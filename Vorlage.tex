% !TeX program = xelatex
\documentclass[10pt]{scrartcl}

% TODO: Metadaten; diese müssen angepasst werden um im Dokument verteilt werden zu können.
\newcommand{\Student}{Vorname Nachname, BSc}
\newcommand{\Matrikelnummer}{9999999999} 
\newcommand{\Geburtstag}{00.00.1000} 
\newcommand{\Wohnort}{0000 Ort} 
\newcommand{\Strasse}{Straße Hausummer}
\newcommand{\Projekttitel}{Titel der Arbeit}
\newcommand{\Projektuntertitel}{Untertitel der Arbeit}
\newcommand{\Studiengang}{Bauingenieurwesen}
\newcommand{\Vertiefungsrichtung}{Entwurf und Konstruktion}
\newcommand{\BetreuerA}{FH-Prof. Dr.-Ing. Vorname Name}
\newcommand{\BetreuerB}{Dr.-Ing. Vorname Name}
\newcommand{\BetreuerC}{~} 
% Wenn Ein Feld nicht benötigt wird, einfach ein geschütztes Leerzeichen, eine Tilde (~) einfügen
%***********************************************************

% Package- and Custom-Commands-Loader

% Paketeinbindung
\usepackage[english, ngerman]{babel}% Sprachpakete für Silbentrennung
\let\latinencoding\relax

\usepackage{fontspec}% Paket zur Einbindung von Schriftarten
\usepackage[headsepline]{scrlayer-scrpage}% Paket für Kopf und Fußzeile
\usepackage{geometry}
\usepackage{titletoc}
\usepackage{lipsum}

% Grafiken
% \usepackage{transparent}
\usepackage{tikz}
\usepackage{pgfplots}
\usepackage{rotating}
\usepackage{document/pgfpie}


\usepackage{subfiles}
\usepackage[pdfencoding=auto,hidelinks,]{hyperref}% Link-Paket für Kapitelverlinkung
\usepackage{appendix}% Anhangpaket
\usepackage{amsmath,amssymb,amstext}% Mathematische Symbolik
\usepackage{mathtools}
\usepackage{graphicx}
\usepackage{tabularx}% Für Tabellen bei denen die X-Umgebung gebraucht wird
\usepackage{booktabs}% schönere Tabellen-Linien
\usepackage{float}
\usepackage{pdfpages}% Einbinden von PDF
\usepackage[%
format=hang, 
labelsep=space, 
labelsep=colon]{caption}% Anpassung der Unterschriften (Bilder/Tabellen)
\usepackage{soul}% Hervorhebung von Text
\usepackage{chemformula}% Für die chemische Formel
\usepackage[onehalfspacing]{setspace}% Anderthalber Zeilenabstand  

% Bibliotheken
\usetikzlibrary{%
	backgrounds,							% Hintergrund
	calc,									% Berechnungen
	positioning,							% relative Positionierungen
	chains, 
	shapes, 
	trees, 
	fadings, 
	external,
	intersections,
	through,
	decorations.pathmorphing,
	arrows,
}
\pgfplotsset{compat=1.15}

% Custom-Commands
\newlength{\tocindent}
\newcommand{\inhaltsverzeichnis}{}
\newcommand{\anhangtrenner}{}
\newcommand{\abbildungsverzeichnis}{}
\newcommand{\tabellenverzeichnis}{}
\newcommand{\anhangverzeichnis}{}
\newcommand{\literaturverzeichnis}{}
\newcommand{\isincluded}{}
\newcommand{\arbeit}{}
\newenvironment{conditions}{}{}
% Laden von Paketen und CustomCommands
\usepackage{document/fh_kaernten_style}% Einbinden der Stildatei
\usepackage{document/fh_kaernten_bibliography}% Stil Literaturverzeichnis

\begin{document}
	
% TODO: Auswahl der Titelseite******************************
% Die Art der Arbeit in den diversen Seiten (z.B. Eidesstattliche Erklärung) wird automatisch angepasst
\newgeometry{left=2.6cm,top=2.52cm,right=2.1cm}
\begin{titlepage}

\begin{minipage}{0.45\textwidth}
\sffamily
\raggedright
\footnotesize
\sodef\myspace{}{.12em}{.5em plus .2em}{2em plus.1em minus.1em}
	\makebox[5.4cm][l]{\myspace{FACHHOCHSCHULE K\"ARNTEN}} \\
\vspace{0.1cm}	
Villacher Straße 1\\
A-9800 Spittal an der Drau\\

\end{minipage}
\raggedleft
\hfill
\begin{minipage}{0.45\textwidth}
\raggedleft	
%\includegraphics[width=4cm]{bilder/fhkaernten}
	\definecolor{FHR}{RGB}{219,35,41}
\definecolor{FHB}{RGB}{0,0,0}
\resizebox{4cm}{!}{%
	\begin{tikzpicture}[yscale=-1, xscale=1,]
	% F
	% F-Dot (1,1)
	\path[fill=FHR] (141.2000,30.7000) .. controls (141.2000,36.0000) and (136.8000,40.4000) .. (131.5000,40.4000) .. controls (126.2000,40.4000) and (121.8000,36.0000) .. (121.8000,30.7000) .. controls (121.8000,25.4000) and (126.2000,21.0000) .. (131.5000,21.0000) .. controls (136.8000,21.0000) and (141.2000,25.4000) .. (141.2000,30.7000);
	% F-Dot (1,2)
	\path[fill=FHR] (165.1000,30.7000) .. controls (165.1000,36.0000) and (160.7000,40.4000) .. (155.4000,40.4000) .. controls (150.1000,40.4000) and (145.7000,36.0000) .. (145.7000,30.7000) .. controls (145.7000,25.4000) and (150.1000,21.0000) .. (155.4000,21.0000) .. controls (160.7000,21.0000) and (165.1000,25.4000) .. (165.1000,30.7000);
	% F-Dot (1,3)
	\path[fill=FHR] (189.0000,30.7000) .. controls (189.0000,36.0000) and (184.6000,40.4000) .. (179.3000,40.4000) .. controls (174.0000,40.4000) and (169.6000,36.0000) .. (169.6000,30.7000) .. controls (169.6000,25.4000) and (174.0000,21.0000) .. (179.3000,21.0000) .. controls (184.7000,21.0000) and (189.0000,25.4000) .. (189.0000,30.7000);
	% F-Dot (2,1)
	\path[fill=FHR] (141.2000,54.6000) .. controls (141.2000,59.9000) and (136.8000,64.3000) .. (131.5000,64.3000) .. controls (126.2000,64.3000) and (121.8000,59.9000) .. (121.8000,54.6000) .. controls (121.8000,49.3000) and (126.2000,44.9000) .. (131.5000,44.9000) .. controls (136.8000,44.9000) and (141.2000,49.3000) .. (141.2000,54.6000);
	% F-Dot (2,2)
	\path[fill=FHR] (165.1000,54.6000) .. controls (165.1000,59.9000) and (160.7000,64.3000) .. (155.4000,64.3000) .. controls (150.1000,64.3000) and (145.7000,59.9000) .. (145.7000,54.6000) .. controls (145.7000,49.3000) and (150.1000,44.9000) .. (155.4000,44.9000) .. controls (160.7000,44.9000) and (165.1000,49.3000) .. (165.1000,54.6000);
	% F-Dot (3,1)
	\path[fill=FHR] (141.2000,78.5000) .. controls (141.2000,83.8000) and (136.8000,88.2000) .. (131.5000,88.2000) .. controls (126.2000,88.2000) and (121.8000,83.8000) .. (121.8000,78.5000) .. controls (121.8000,73.2000) and (126.2000,68.8000) .. (131.5000,68.8000) .. controls (136.8000,68.8000) and (141.2000,73.2000) .. (141.2000,78.5000);
	% H
	% H-Dot (1,1)
	\path[fill=FHB] (222.6000,30.7000) .. controls (222.6000,36.0000) and (218.2000,40.4000) .. (212.9000,40.4000) .. controls (207.6000,40.4000) and (203.2000,36.0000) .. (203.2000,30.7000) .. controls (203.2000,25.4000) and (207.6000,21.0000) .. (212.9000,21.0000) .. controls (218.2000,21.0000) and (222.6000,25.4000) .. (222.6000,30.7000);
	% H-Dot (1,3)
	\path[fill=FHB] (270.4000,30.7000) .. controls (270.4000,36.0000) and (266.0000,40.4000) .. (260.7000,40.4000) .. controls (255.4000,40.4000) and (251.0000,36.0000) .. (251.0000,30.7000) .. controls (251.0000,25.4000) and (255.4000,21.0000) .. (260.7000,21.0000) .. controls (266.0000,21.0000) and (270.4000,25.4000) .. (270.4000,30.7000);
	% H-Dot (2,1)
	\path[fill=FHB] (222.6000,54.6000) .. controls (222.6000,59.9000) and (218.2000,64.3000) .. (212.9000,64.3000) .. controls (207.6000,64.3000) and (203.2000,59.9000) .. (203.2000,54.6000) .. controls (203.2000,49.3000) and (207.6000,44.9000) .. (212.9000,44.9000) .. controls (218.2000,44.9000) and (222.6000,49.3000) .. (222.6000,54.6000);
	% H-Dot (2,2)
	\path[fill=FHB] (246.5000,54.6000) .. controls (246.5000,59.9000) and (242.1000,64.3000) .. (236.8000,64.3000) .. controls (231.5000,64.3000) and (227.1000,59.9000) .. (227.1000,54.6000) .. controls (227.1000,49.3000) and (231.5000,44.9000) .. (236.8000,44.9000) .. controls (242.1000,44.9000) and (246.5000,49.3000) .. (246.5000,54.6000);
	% H-Dot (2,3)
	\path[fill=FHB] (270.4000,54.6000) .. controls (270.4000,59.9000) and (266.0000,64.3000) .. (260.7000,64.3000) .. controls (255.4000,64.3000) and (251.0000,59.9000) .. (251.0000,54.6000) .. controls (251.0000,49.3000) and (255.4000,44.9000) .. (260.7000,44.9000) .. controls (266.0000,44.9000) and (270.4000,49.3000) .. (270.4000,54.6000);
	% H-Dot (3,1)
	\path[fill=FHB] (222.6000,78.5000) .. controls (222.6000,83.8000) and (218.2000,88.2000) .. (212.9000,88.2000) .. controls (207.6000,88.2000) and (203.2000,83.8000) .. (203.2000,78.5000) .. controls (203.2000,73.2000) and (207.6000,68.8000) .. (212.9000,68.8000) .. controls (218.2000,68.8000) and (222.6000,73.2000) .. (222.6000,78.5000);
	% H-Dot (3,3)
	\path[fill=FHB] (270.4000,78.5000) .. controls (270.4000,83.8000) and (266.0000,88.2000) .. (260.7000,88.2000) .. controls (255.4000,88.2000) and (251.0000,83.8000) .. (251.0000,78.5000) .. controls (251.0000,73.2000) and (255.4000,68.8000) .. (260.7000,68.8000) .. controls (266.0000,68.8000) and (270.4000,73.2000) .. (270.4000,78.5000);
	%-------------------------------------------------------------------------------------------
	% Fachhochschule
	\path[shift={(-102.4,-398.0)},fill=FHB] (315.7000,400.2000) -- (312.5000,400.2000) -- (312.5000,402.1000) -- (315.7000,402.1000) -- (315.7000,404.0000) -- (312.5000,404.0000) -- (312.5000,408.7000) -- (310.6000,408.7000) -- (310.6000,398.3000) -- (315.7000,398.3000) -- cycle;
	% fAchhochschule
	\path[fill=FHB] (221.7000,0.3000) -- (223.7000,0.3000) -- (227.7000,10.7000) -- (225.6000,10.7000) -- (224.8000,8.6000) -- (220.6000,8.6000) -- (219.8000,10.7000) -- (217.7000,10.7000) -- cycle(222.7000,3.1000) -- (221.3000,6.7000) -- (224.1000,6.7000) -- cycle;
	% faChhochschule
	\path[fill=FHB] (242.2000,2.1000) -- (240.8000,3.4000) .. controls (239.9000,2.4000) and (238.8000,1.9000) .. (237.6000,1.9000) .. controls (236.6000,1.9000) and (235.8000,2.2000) .. (235.1000,2.9000) .. controls (234.4000,3.6000) and (234.1000,4.4000) .. (234.1000,5.4000) .. controls (234.1000,6.1000) and (234.2000,6.7000) .. (234.5000,7.2000) .. controls (234.8000,7.7000) and (235.2000,8.2000) .. (235.8000,8.5000) .. controls (236.3000,8.8000) and (237.0000,9.0000) .. (237.6000,9.0000) .. controls (238.2000,9.0000) and (238.7000,8.9000) .. (239.2000,8.7000) .. controls (239.7000,8.5000) and (240.2000,8.1000) .. (240.8000,7.5000) -- (242.1000,8.9000) .. controls (241.3000,9.7000) and (240.6000,10.2000) .. (239.9000,10.5000) .. controls (239.2000,10.8000) and (238.4000,10.9000) .. (237.6000,10.9000) .. controls (236.0000,10.9000) and (234.7000,10.4000) .. (233.6000,9.4000) .. controls (232.6000,8.4000) and (232.1000,7.1000) .. (232.1000,5.5000) .. controls (232.1000,4.5000) and (232.3000,3.5000) .. (232.8000,2.7000) .. controls (233.3000,1.9000) and (233.9000,1.3000) .. (234.8000,0.8000) .. controls (235.7000,0.3000) and (236.6000,0.1000) .. (237.6000,0.1000) .. controls (238.5000,0.1000) and (239.3000,0.3000) .. (240.1000,0.6000) .. controls (240.9000,0.9000) and (241.6000,1.5000) .. (242.2000,2.1000);
	% facHhochschule
	\path[shift={(-102.4,-398.0)},fill=FHB] (351.8000,402.2000) -- (355.2000,402.2000) -- (355.2000,398.3000) -- (357.2000,398.3000) -- (357.2000,408.7000) -- (355.2000,408.7000) -- (355.2000,404.1000) -- (351.8000,404.1000) -- (351.8000,408.7000) -- (349.7000,408.7000) -- (349.7000,398.3000) -- (351.8000,398.3000) -- cycle;
	% fachHochschule
	\path[shift={(-102.4,-398.0)},fill=FHB] (365.0000,402.2000) -- (368.5000,402.2000) -- (368.5000,398.3000) -- (370.5000,398.3000) -- (370.5000,408.7000) -- (368.5000,408.7000) -- (368.5000,404.1000) -- (365.0000,404.1000) -- (365.0000,408.7000) -- (363.0000,408.7000) -- (363.0000,398.3000) -- (365.0000,398.3000) -- cycle;
	% fachhOchschule
	\path[fill=FHB] (278.8000,0.0000) .. controls (280.3000,0.0000) and (281.5000,0.5000) .. (282.6000,1.6000) .. controls (283.7000,2.7000) and (284.2000,4.0000) .. (284.2000,5.5000) .. controls (284.2000,7.0000) and (283.7000,8.3000) .. (282.6000,9.3000) .. controls (281.6000,10.3000) and (280.3000,10.9000) .. (278.8000,10.9000) .. controls (277.2000,10.9000) and (276.0000,10.4000) .. (274.9000,9.3000) .. controls (273.9000,8.2000) and (273.4000,6.9000) .. (273.4000,5.5000) .. controls (273.4000,4.5000) and (273.6000,3.6000) .. (274.1000,2.8000) .. controls (274.6000,2.0000) and (275.2000,1.3000) .. (276.1000,0.8000) .. controls (276.9000,0.3000) and (277.8000,0.0000) .. (278.8000,0.0000)(278.8000,2.0000) .. controls (277.8000,2.0000) and (277.0000,2.3000) .. (276.4000,3.0000) .. controls (275.7000,3.7000) and (275.4000,4.5000) .. (275.4000,5.5000) .. controls (275.4000,6.6000) and (275.8000,7.6000) .. (276.6000,8.2000) .. controls (277.2000,8.7000) and (278.0000,9.0000) .. (278.8000,9.0000) .. controls (279.7000,9.0000) and (280.5000,8.7000) .. (281.2000,8.0000) .. controls (281.9000,7.3000) and (282.2000,6.5000) .. (282.2000,5.5000) .. controls (282.2000,4.5000) and (281.9000,3.7000) .. (281.2000,3.0000) .. controls (280.5000,2.3000) and (279.7000,2.0000) .. (278.8000,2.0000);
	% fachhoChschule
	\path[fill=FHB] (299.0000,2.1000) -- (297.6000,3.4000) .. controls (296.7000,2.4000) and (295.6000,1.9000) .. (294.4000,1.9000) .. controls (293.4000,1.9000) and (292.6000,2.2000) .. (291.9000,2.9000) .. controls (291.2000,3.6000) and (290.9000,4.4000) .. (290.9000,5.4000) .. controls (290.9000,6.1000) and (291.0000,6.7000) .. (291.3000,7.2000) .. controls (291.6000,7.7000) and (292.0000,8.2000) .. (292.6000,8.5000) .. controls (293.2000,8.8000) and (293.8000,9.0000) .. (294.4000,9.0000) .. controls (295.0000,9.0000) and (295.5000,8.9000) .. (296.0000,8.7000) .. controls (296.5000,8.5000) and (297.0000,8.1000) .. (297.6000,7.5000) -- (298.9000,8.9000) .. controls (298.1000,9.7000) and (297.4000,10.2000) .. (296.7000,10.5000) .. controls (296.0000,10.8000) and (295.2000,10.9000) .. (294.4000,10.9000) .. controls (292.8000,10.9000) and (291.5000,10.4000) .. (290.4000,9.4000) .. controls (289.4000,8.4000) and (288.9000,7.1000) .. (288.9000,5.5000) .. controls (288.9000,4.5000) and (289.1000,3.5000) .. (289.6000,2.7000) .. controls (290.1000,1.9000) and (290.7000,1.3000) .. (291.6000,0.8000) .. controls (292.5000,0.3000) and (293.4000,0.1000) .. (294.4000,0.1000) .. controls (295.3000,0.1000) and (296.1000,0.3000) .. (296.9000,0.6000) .. controls (297.8000,0.9000) and (298.5000,1.5000) .. (299.0000,2.1000);
	% fachhocHschule
	\path[shift={(-102.4,-398.0)},fill=FHB] (408.6000,402.2000) -- (412.1000,402.2000) -- (412.1000,398.3000) -- (414.1000,398.3000) -- (414.1000,408.7000) -- (412.1000,408.7000) -- (412.1000,404.1000) -- (408.6000,404.1000) -- (408.6000,408.7000) -- (406.6000,408.7000) -- (406.6000,398.3000) -- (408.6000,398.3000) -- cycle;
	% fachhochSchule
	\path[fill=FHB] (322.9000,1.7000) -- (321.4000,3.0000) .. controls (320.9000,2.3000) and (320.4000,1.9000) .. (319.8000,1.9000) .. controls (319.5000,1.9000) and (319.3000,2.0000) .. (319.2000,2.1000) .. controls (319.0000,2.2000) and (319.0000,2.4000) .. (319.0000,2.6000) .. controls (319.0000,2.8000) and (319.1000,2.9000) .. (319.2000,3.1000) .. controls (319.4000,3.3000) and (319.8000,3.8000) .. (320.7000,4.4000) .. controls (321.5000,5.0000) and (321.9000,5.4000) .. (322.1000,5.6000) .. controls (322.5000,6.0000) and (322.8000,6.4000) .. (323.0000,6.8000) .. controls (323.2000,7.2000) and (323.2000,7.6000) .. (323.2000,8.0000) .. controls (323.2000,8.9000) and (322.9000,9.6000) .. (322.3000,10.1000) .. controls (321.7000,10.7000) and (320.9000,10.9000) .. (320.0000,10.9000) .. controls (319.3000,10.9000) and (318.6000,10.7000) .. (318.1000,10.4000) .. controls (317.6000,10.0000) and (317.1000,9.5000) .. (316.7000,8.7000) -- (318.4000,7.7000) .. controls (318.9000,8.6000) and (319.5000,9.1000) .. (320.1000,9.1000) .. controls (320.4000,9.1000) and (320.7000,9.0000) .. (321.0000,8.8000) .. controls (321.2000,8.6000) and (321.3000,8.4000) .. (321.3000,8.1000) .. controls (321.3000,7.9000) and (321.2000,7.6000) .. (321.0000,7.4000) .. controls (320.8000,7.2000) and (320.4000,6.8000) .. (319.8000,6.3000) .. controls (318.7000,5.4000) and (317.9000,4.7000) .. (317.6000,4.2000) .. controls (317.3000,3.7000) and (317.1000,3.2000) .. (317.1000,2.7000) .. controls (317.1000,2.0000) and (317.4000,1.4000) .. (317.9000,0.8000) .. controls (318.5000,0.3000) and (319.1000,0.0000) .. (319.9000,0.0000) .. controls (320.4000,0.0000) and (320.9000,0.1000) .. (321.4000,0.4000) .. controls (321.9000,0.6000) and (322.4000,1.1000) .. (322.9000,1.7000);
	% fachhochsChule
	\path[fill=FHB] (338.0000,2.1000) -- (336.6000,3.4000) .. controls (335.7000,2.4000) and (334.6000,1.9000) .. (333.4000,1.9000) .. controls (332.4000,1.9000) and (331.6000,2.2000) .. (330.9000,2.9000) .. controls (330.2000,3.6000) and (329.9000,4.4000) .. (329.9000,5.4000) .. controls (329.9000,6.1000) and (330.1000,6.7000) .. (330.4000,7.2000) .. controls (330.7000,7.7000) and (331.1000,8.2000) .. (331.7000,8.5000) .. controls (332.2000,8.8000) and (332.9000,9.0000) .. (333.5000,9.0000) .. controls (334.1000,9.0000) and (334.6000,8.9000) .. (335.1000,8.7000) .. controls (335.6000,8.5000) and (336.1000,8.1000) .. (336.7000,7.5000) -- (338.0000,8.9000) .. controls (337.2000,9.7000) and (336.5000,10.2000) .. (335.8000,10.5000) .. controls (335.1000,10.8000) and (334.3000,10.9000) .. (333.5000,10.9000) .. controls (331.9000,10.9000) and (330.6000,10.4000) .. (329.5000,9.4000) .. controls (328.5000,8.4000) and (328.0000,7.1000) .. (328.0000,5.5000) .. controls (328.0000,4.5000) and (328.2000,3.5000) .. (328.7000,2.7000) .. controls (329.2000,1.9000) and (329.8000,1.3000) .. (330.7000,0.8000) .. controls (331.6000,0.3000) and (332.5000,0.1000) .. (333.5000,0.1000) .. controls (334.4000,0.1000) and (335.2000,0.3000) .. (336.0000,0.6000) .. controls (336.7000,0.9000) and (337.4000,1.5000) .. (338.0000,2.1000);
	% fachhochscHule
	\path[shift={(-102.4,-398.0)},fill=FHB] (447.6000,402.2000) -- (451.0000,402.2000) -- (451.0000,398.3000) -- (453.0000,398.3000) -- (453.0000,408.7000) -- (451.0000,408.7000) -- (451.0000,404.1000) -- (447.6000,404.1000) -- (447.6000,408.7000) -- (445.6000,408.7000) -- (445.6000,398.3000) -- (447.6000,398.3000) -- cycle;
	% fachhochschUle
	\path[fill=FHB] (356.4000,0.3000) -- (358.4000,0.3000) -- (358.4000,7.0000) .. controls (358.4000,7.6000) and (358.5000,8.0000) .. (358.6000,8.2000) .. controls (358.7000,8.4000) and (358.9000,8.6000) .. (359.1000,8.8000) .. controls (359.3000,8.9000) and (359.6000,9.0000) .. (359.9000,9.0000) .. controls (360.3000,9.0000) and (360.6000,8.9000) .. (360.8000,8.8000) .. controls (361.1000,8.6000) and (361.2000,8.4000) .. (361.3000,8.2000) .. controls (361.4000,8.0000) and (361.4000,7.5000) .. (361.4000,6.8000) -- (361.4000,0.3000) -- (363.4000,0.3000) -- (363.4000,6.5000) .. controls (363.4000,7.5000) and (363.3000,8.3000) .. (363.2000,8.7000) .. controls (363.1000,9.1000) and (362.9000,9.5000) .. (362.5000,9.9000) .. controls (362.2000,10.3000) and (361.8000,10.6000) .. (361.4000,10.7000) .. controls (361.0000,10.9000) and (360.5000,11.0000) .. (359.9000,11.0000) .. controls (359.1000,11.0000) and (358.5000,10.8000) .. (357.9000,10.5000) .. controls (357.3000,10.2000) and (356.9000,9.7000) .. (356.7000,9.2000) .. controls (356.5000,8.7000) and (356.3000,7.8000) .. (356.3000,6.6000) -- (356.3000,0.3000) -- cycle;
	% fachhochschuLe
	\path[shift={(-102.4,-398.0)},fill=FHB] (473.5000,406.8000) -- (476.4000,406.8000) -- (476.4000,408.7000) -- (471.5000,408.7000) -- (471.5000,398.3000) -- (473.5000,398.3000) -- cycle;
	% fachhochschulE
	\path[shift={(-102.4,-398.0)},fill=FHB] (487.0000,400.2000) -- (483.3000,400.2000) -- (483.3000,402.1000) -- (487.0000,402.1000) -- (487.0000,404.0000) -- (483.3000,404.0000) -- (483.3000,406.7000) -- (487.0000,406.7000) -- (487.0000,408.7000) -- (481.4000,408.7000) -- (481.4000,398.3000) -- (487.0000,398.3000) -- cycle;
	%-------------------------------------------------------------------------------------------
	% Kärnten
	\path[shift={(-102.4,-398.0)},fill=FHR] (390.6000,425.4000) -- (394.4000,420.5000) -- (397.7000,420.5000) -- (392.7000,426.9000) -- (398.2000,434.8000) -- (395.0000,434.8000) -- (390.6000,428.4000) -- (390.6000,434.8000) -- (387.9000,434.8000) -- (387.9000,420.5000) -- (390.6000,420.5000) -- cycle;
	% kÄrnten
	\path[fill=FHR] (304.8000,22.5000) -- (307.5000,22.5000) -- (313.0000,36.7000) -- (310.2000,36.7000) -- (309.1000,33.8000) -- (303.3000,33.8000) -- (302.1000,36.7000) -- (299.3000,36.7000) -- cycle(304.0000,18.7000) .. controls (304.4000,18.7000) and (304.7000,18.8000) .. (305.0000,19.1000) .. controls (305.3000,19.4000) and (305.4000,19.7000) .. (305.4000,20.1000) .. controls (305.4000,20.5000) and (305.3000,20.8000) .. (305.0000,21.1000) .. controls (304.7000,21.4000) and (304.4000,21.5000) .. (304.0000,21.5000) .. controls (303.6000,21.5000) and (303.3000,21.4000) .. (303.0000,21.1000) .. controls (302.7000,20.8000) and (302.6000,20.5000) .. (302.6000,20.1000) .. controls (302.6000,19.7000) and (302.7000,19.4000) .. (303.0000,19.1000) .. controls (303.3000,18.8000) and (303.7000,18.7000) .. (304.0000,18.7000)(306.2000,26.3000) -- (304.3000,31.2000) -- (308.1000,31.2000) -- cycle(308.3000,18.7000) .. controls (308.7000,18.7000) and (309.0000,18.8000) .. (309.3000,19.1000) .. controls (309.6000,19.4000) and (309.7000,19.7000) .. (309.7000,20.1000) .. controls (309.7000,20.5000) and (309.6000,20.8000) .. (309.3000,21.1000) .. controls (309.0000,21.4000) and (308.7000,21.5000) .. (308.3000,21.5000) .. controls (307.9000,21.5000) and (307.6000,21.4000) .. (307.3000,21.1000) .. controls (307.0000,20.8000) and (306.9000,20.5000) .. (306.9000,20.1000) .. controls (306.9000,19.7000) and (307.0000,19.4000) .. (307.3000,19.1000) .. controls (307.6000,18.8000) and (307.9000,18.7000) .. (308.3000,18.7000);
	% käRnten
	\path[fill=FHR] (318.1000,22.5000) -- (321.0000,22.5000) .. controls (322.6000,22.5000) and (323.7000,22.6000) .. (324.4000,22.9000) .. controls (325.1000,23.2000) and (325.6000,23.6000) .. (326.0000,24.3000) .. controls (326.4000,25.0000) and (326.6000,25.7000) .. (326.6000,26.6000) .. controls (326.6000,27.5000) and (326.4000,28.3000) .. (325.9000,28.9000) .. controls (325.5000,29.5000) and (324.8000,30.0000) .. (323.9000,30.3000) -- (327.3000,36.6000) -- (324.3000,36.6000) -- (321.1000,30.6000) -- (320.8000,30.6000) -- (320.8000,36.6000) -- (318.1000,36.6000) -- cycle(320.8000,28.1000) -- (321.7000,28.1000) .. controls (322.6000,28.1000) and (323.2000,28.0000) .. (323.5000,27.8000) .. controls (323.8000,27.6000) and (324.0000,27.2000) .. (324.0000,26.7000) .. controls (324.0000,26.4000) and (323.9000,26.1000) .. (323.8000,25.9000) .. controls (323.6000,25.7000) and (323.4000,25.5000) .. (323.1000,25.4000) .. controls (322.8000,25.3000) and (322.3000,25.2000) .. (321.6000,25.2000) -- (320.8000,25.2000) -- cycle;
	% kärNten
	\path[shift={(-102.4,-398.0)},fill=FHR] (443.2000,429.9000) -- (443.2000,420.5000) -- (445.9000,420.5000) -- (445.9000,434.8000) -- (443.3000,434.8000) -- (437.2000,425.4000) -- (437.2000,434.8000) -- (434.5000,434.8000) -- (434.5000,420.5000) -- (437.1000,420.5000) -- cycle;
	% kärnTen
	\path[shift={(-102.4,-398.0)},fill=FHR] (458.4000,423.2000) -- (455.8000,423.2000) -- (455.8000,434.8000) -- (453.0000,434.8000) -- (453.0000,423.2000) -- (450.5000,423.2000) -- (450.5000,420.5000) -- (458.4000,420.5000) -- cycle;
	% kärntEn
	\path[shift={(-102.4,-398.0)},fill=FHR] (470.8000,423.2000) -- (465.7000,423.2000) -- (465.7000,425.8000) -- (470.8000,425.8000) -- (470.8000,428.4000) -- (465.7000,428.4000) -- (465.7000,432.1000) -- (470.8000,432.1000) -- (470.8000,434.8000) -- (463.0000,434.8000) -- (463.0000,420.5000) -- (470.8000,420.5000) -- cycle;
	% kärnteN
	\path[shift={(-102.4,-398.0)},fill=FHR] (484.7000,429.9000) -- (484.7000,420.5000) -- (487.4000,420.5000) -- (487.4000,434.8000) -- (484.8000,434.8000) -- (478.8000,425.4000) -- (478.8000,434.8000) -- (476.1000,434.8000) -- (476.1000,420.5000) -- (478.7000,420.5000) -- cycle;
	%-------------------------------------------------------------------------------------------
	% Carinthia
	\path[fill=FHR] (10.7000,25.3000) -- (9.3000,26.6000) .. controls (8.3000,25.6000) and (7.3000,25.1000) .. (6.1000,25.1000) .. controls (5.1000,25.1000) and (4.2000,25.4000) .. (3.5000,26.1000) .. controls (2.8000,26.8000) and (2.5000,27.6000) .. (2.5000,28.7000) .. controls (2.5000,29.4000) and (2.7000,30.0000) .. (3.0000,30.6000) .. controls (3.3000,31.2000) and (3.7000,31.6000) .. (4.3000,31.9000) .. controls (4.9000,32.2000) and (5.5000,32.4000) .. (6.2000,32.4000) .. controls (6.8000,32.4000) and (7.3000,32.3000) .. (7.8000,32.1000) .. controls (8.3000,31.9000) and (8.8000,31.5000) .. (9.4000,30.9000) -- (10.8000,32.3000) .. controls (10.0000,33.1000) and (9.3000,33.6000) .. (8.6000,33.9000) .. controls (7.9000,34.2000) and (7.1000,34.3000) .. (6.2000,34.3000) .. controls (4.5000,34.3000) and (3.2000,33.8000) .. (2.1000,32.7000) .. controls (1.0000,31.7000) and (0.5000,30.3000) .. (0.5000,28.7000) .. controls (0.5000,27.6000) and (0.7000,26.7000) .. (1.2000,25.9000) .. controls (1.7000,25.1000) and (2.4000,24.4000) .. (3.3000,23.9000) .. controls (4.2000,23.4000) and (5.2000,23.1000) .. (6.2000,23.1000) .. controls (7.1000,23.1000) and (7.9000,23.3000) .. (8.7000,23.7000) .. controls (9.4000,24.1000) and (10.1000,24.6000) .. (10.7000,25.3000);
	% cArinthia
	\path[fill=FHR] (17.5000,23.4000) -- (19.6000,23.4000) -- (23.6000,34.0000) -- (21.5000,34.0000) -- (20.7000,31.8000) -- (16.4000,31.8000) -- (15.5000,34.0000) -- (13.4000,34.0000) -- cycle(18.5000,26.2000) -- (17.1000,29.8000) -- (19.9000,29.8000) -- cycle;
	% caRinthia
	\path[fill=FHR] (27.1000,23.4000) -- (29.2000,23.4000) .. controls (30.4000,23.4000) and (31.2000,23.5000) .. (31.7000,23.7000) .. controls (32.2000,23.9000) and (32.6000,24.3000) .. (32.9000,24.7000) .. controls (33.2000,25.2000) and (33.4000,25.8000) .. (33.4000,26.4000) .. controls (33.4000,27.1000) and (33.2000,27.7000) .. (32.9000,28.2000) .. controls (32.6000,28.7000) and (32.1000,29.0000) .. (31.4000,29.3000) -- (33.9000,34.0000) -- (31.7000,34.0000) -- (29.3000,29.5000) -- (29.1000,29.5000) -- (29.1000,34.0000) -- (27.1000,34.0000) -- cycle(29.1000,27.5000) -- (29.7000,27.5000) .. controls (30.3000,27.5000) and (30.8000,27.4000) .. (31.0000,27.2000) .. controls (31.2000,27.0000) and (31.4000,26.8000) .. (31.4000,26.4000) .. controls (31.4000,26.2000) and (31.3000,26.0000) .. (31.2000,25.8000) .. controls (31.1000,25.6000) and (30.9000,25.5000) .. (30.7000,25.4000) .. controls (30.5000,25.3000) and (30.1000,25.3000) .. (29.6000,25.3000) -- (29.0000,25.3000) -- (29.0000,27.5000) -- cycle;
	% carInthia
	\path[fill=FHR] (37.2000,23.4000) rectangle (39.2000,34.0000);
	% cariNthia
	\path[shift={(-102.4,-398.0)},fill=FHR] (152.1000,428.4000) -- (152.1000,421.4000) -- (154.1000,421.4000) -- (154.1000,432.0000) -- (152.2000,432.0000) -- (147.7000,425.0000) -- (147.7000,432.0000) -- (145.6000,432.0000) -- (145.6000,421.4000) -- (147.6000,421.4000) -- cycle;
	% carinThia
	\path[shift={(-102.4,-398.0)},fill=FHR] (163.2000,423.4000) -- (161.2000,423.4000) -- (161.2000,432.0000) -- (159.2000,432.0000) -- (159.2000,423.4000) -- (157.3000,423.4000) -- (157.3000,421.4000) -- (163.2000,421.4000) -- cycle;
	% carintHia
	\path[shift={(-102.4,-398.0)},fill=FHR] (168.4000,425.4000) -- (171.9000,425.4000) -- (171.9000,421.4000) -- (174.0000,421.4000) -- (174.0000,432.0000) -- (171.9000,432.0000) -- (171.9000,427.4000) -- (168.4000,427.4000) -- (168.4000,432.0000) -- (166.3000,432.0000) -- (166.3000,421.4000) -- (168.4000,421.4000) -- cycle;
	% carinthIa
	\path[fill=FHR] (75.6000,23.4000) rectangle (77.6000,34.0000);
	% carinthiA
	\path[fill=FHR] (84.9000,23.4000) -- (87.0000,23.4000) -- (91.1000,34.0000) -- (89.0000,34.0000) -- (88.2000,31.8000) -- (83.9000,31.8000) -- (83.0000,34.0000) -- (80.9000,34.0000) -- cycle(86.0000,26.2000) -- (84.6000,29.8000) -- (87.4000,29.8000) -- cycle;
	%-------------------------------------------------------------------------------------------
	% University
	\path[fill=FHB] (0.8000,41.1000) -- (2.8000,41.1000) -- (2.8000,48.0000) .. controls (2.8000,48.6000) and (2.9000,49.0000) .. (3.0000,49.3000) .. controls (3.1000,49.5000) and (3.3000,49.7000) .. (3.5000,49.9000) .. controls (3.7000,50.0000) and (4.0000,50.1000) .. (4.4000,50.1000) .. controls (4.8000,50.1000) and (5.1000,50.0000) .. (5.3000,49.9000) .. controls (5.6000,49.7000) and (5.7000,49.5000) .. (5.8000,49.3000) .. controls (5.9000,49.0000) and (5.9000,48.6000) .. (5.9000,47.9000) -- (5.9000,41.3000) -- (7.9000,41.3000) -- (7.9000,47.6000) .. controls (7.9000,48.7000) and (7.8000,49.4000) .. (7.7000,49.8000) .. controls (7.6000,50.2000) and (7.3000,50.6000) .. (7.0000,51.0000) .. controls (6.7000,51.4000) and (6.3000,51.7000) .. (5.8000,51.9000) .. controls (5.4000,52.1000) and (4.8000,52.2000) .. (4.3000,52.2000) .. controls (3.5000,52.2000) and (2.9000,52.0000) .. (2.3000,51.7000) .. controls (1.7000,51.3000) and (1.3000,50.9000) .. (1.1000,50.4000) .. controls (0.9000,49.9000) and (0.7000,49.0000) .. (0.7000,47.7000) -- (0.7000,41.1000) -- cycle;
	% uNiversity
	\path[shift={(-102.4,-398.0)},fill=FHB] (120.8000,446.1000) -- (120.8000,439.1000) -- (122.8000,439.1000) -- (122.8000,449.8000) -- (120.8000,449.8000) -- (116.3000,442.8000) -- (116.3000,449.8000) -- (114.3000,449.8000) -- (114.3000,439.1000) -- (116.2000,439.1000) -- cycle;
	% unIversity
	\path[fill=FHB] (24.1000,41.1000) rectangle (26.1000,51.7000);
	% uniVersity
	\path[shift={(-102.4,-398.0)},fill=FHB] (136.2000,446.7000) -- (139.0000,439.1000) -- (141.0000,439.1000) -- (137.2000,449.8000) -- (135.3000,449.8000) -- (131.5000,439.1000) -- (133.6000,439.1000) -- cycle;
	% univErsity
	\path[shift={(-102.4,-398.0)},fill=FHB] (149.9000,441.1000) -- (146.2000,441.1000) -- (146.2000,443.0000) -- (149.9000,443.0000) -- (149.9000,445.0000) -- (146.2000,445.0000) -- (146.2000,447.8000) -- (149.9000,447.8000) -- (149.9000,449.8000) -- (144.1000,449.8000) -- (144.1000,439.1000) -- (149.9000,439.1000) -- cycle;
	% univeRsity
	\path[fill=FHB] (51.0000,41.1000) -- (53.1000,41.1000) .. controls (54.3000,41.1000) and (55.1000,41.2000) .. (55.6000,41.4000) .. controls (56.1000,41.6000) and (56.5000,42.0000) .. (56.8000,42.4000) .. controls (57.1000,42.9000) and (57.3000,43.5000) .. (57.3000,44.1000) .. controls (57.3000,44.8000) and (57.1000,45.4000) .. (56.8000,45.9000) .. controls (56.5000,46.4000) and (56.0000,46.7000) .. (55.3000,47.0000) -- (57.8000,51.7000) -- (55.6000,51.7000) -- (53.2000,47.2000) -- (53.0000,47.2000) -- (53.0000,51.7000) -- (51.0000,51.7000) -- cycle(53.1000,45.3000) -- (53.7000,45.3000) .. controls (54.3000,45.3000) and (54.8000,45.2000) .. (55.0000,45.0000) .. controls (55.2000,44.8000) and (55.4000,44.6000) .. (55.4000,44.2000) .. controls (55.4000,44.0000) and (55.3000,43.8000) .. (55.2000,43.6000) .. controls (55.1000,43.4000) and (54.9000,43.3000) .. (54.7000,43.2000) .. controls (54.5000,43.1000) and (54.1000,43.1000) .. (53.6000,43.1000) -- (53.0000,43.1000) -- (53.0000,45.3000) -- cycle;
	% univerSity
	\path[fill=FHB] (66.5000,42.6000) -- (65.0000,43.9000) .. controls (64.5000,43.2000) and (63.9000,42.8000) .. (63.4000,42.8000) .. controls (63.1000,42.8000) and (62.9000,42.9000) .. (62.7000,43.0000) .. controls (62.5000,43.1000) and (62.4000,43.3000) .. (62.4000,43.5000) .. controls (62.4000,43.7000) and (62.5000,43.8000) .. (62.6000,44.0000) .. controls (62.8000,44.2000) and (63.3000,44.7000) .. (64.1000,45.4000) .. controls (64.9000,46.0000) and (65.3000,46.5000) .. (65.5000,46.6000) .. controls (65.9000,47.0000) and (66.2000,47.4000) .. (66.4000,47.8000) .. controls (66.6000,48.2000) and (66.7000,48.6000) .. (66.7000,49.0000) .. controls (66.7000,49.9000) and (66.4000,50.6000) .. (65.8000,51.2000) .. controls (65.2000,51.8000) and (64.4000,52.1000) .. (63.4000,52.1000) .. controls (62.6000,52.1000) and (62.0000,51.9000) .. (61.4000,51.5000) .. controls (60.8000,51.1000) and (60.4000,50.5000) .. (60.0000,49.8000) -- (61.7000,48.8000) .. controls (62.2000,49.7000) and (62.8000,50.2000) .. (63.5000,50.2000) .. controls (63.8000,50.2000) and (64.1000,50.1000) .. (64.4000,49.9000) .. controls (64.6000,49.7000) and (64.8000,49.5000) .. (64.8000,49.2000) .. controls (64.8000,49.0000) and (64.7000,48.7000) .. (64.5000,48.5000) .. controls (64.3000,48.3000) and (63.9000,47.9000) .. (63.3000,47.4000) .. controls (62.1000,46.5000) and (61.4000,45.7000) .. (61.1000,45.2000) .. controls (60.8000,44.7000) and (60.6000,44.2000) .. (60.6000,43.7000) .. controls (60.6000,43.0000) and (60.9000,42.3000) .. (61.4000,41.8000) .. controls (62.0000,41.3000) and (62.7000,41.0000) .. (63.5000,41.0000) .. controls (64.0000,41.0000) and (64.5000,41.1000) .. (65.0000,41.4000) .. controls (65.5000,41.5000) and (66.0000,41.9000) .. (66.5000,42.6000);
	% universIty
	\path[fill=FHB] (70.0000,41.1000) rectangle (72.0000,51.7000);
	% universiTy
	\path[shift={(-102.4,-398.0)},fill=FHB] (183.0000,441.1000) -- (181.0000,441.1000) -- (181.0000,449.8000) -- (179.0000,449.8000) -- (179.0000,441.1000) -- (177.1000,441.1000) -- (177.1000,439.1000) -- (183.0000,439.1000) -- cycle;
	% universitY
	\path[shift={(-102.4,-398.0)},fill=FHB] (189.2000,443.0000) -- (191.2000,439.1000) -- (193.4000,439.1000) -- (190.3000,445.2000) -- (190.3000,449.8000) -- (188.2000,449.8000) -- (188.2000,445.2000) -- (185.1000,439.1000) -- (187.3000,439.1000) -- cycle;
	%-------------------------------------------------------------------------------------------
	% Of
	\path[fill=FHB] (7.4000,58.6000) .. controls (8.9000,58.6000) and (10.2000,59.1000) .. (11.3000,60.2000) .. controls (12.4000,61.3000) and (12.9000,62.6000) .. (12.9000,64.2000) .. controls (12.9000,65.7000) and (12.4000,67.1000) .. (11.3000,68.1000) .. controls (10.2000,69.2000) and (8.9000,69.7000) .. (7.4000,69.7000) .. controls (5.8000,69.7000) and (4.5000,69.2000) .. (3.4000,68.0000) .. controls (2.3000,66.9000) and (1.8000,65.6000) .. (1.8000,64.1000) .. controls (1.8000,63.1000) and (2.0000,62.2000) .. (2.5000,61.3000) .. controls (3.0000,60.4000) and (3.7000,59.8000) .. (4.5000,59.3000) .. controls (5.5000,58.9000) and (6.4000,58.6000) .. (7.4000,58.6000)(7.4000,60.6000) .. controls (6.4000,60.6000) and (5.6000,60.9000) .. (4.9000,61.6000) .. controls (4.2000,62.3000) and (3.9000,63.2000) .. (3.9000,64.2000) .. controls (3.9000,65.4000) and (4.3000,66.3000) .. (5.2000,67.0000) .. controls (5.9000,67.5000) and (6.6000,67.8000) .. (7.5000,67.8000) .. controls (8.5000,67.8000) and (9.3000,67.5000) .. (10.0000,66.8000) .. controls (10.7000,66.1000) and (11.0000,65.3000) .. (11.0000,64.2000) .. controls (11.0000,63.2000) and (10.7000,62.3000) .. (10.0000,61.6000) .. controls (9.2000,61.0000) and (8.4000,60.6000) .. (7.4000,60.6000);
	% oF
	\path[shift={(-102.4,-398.0)},fill=FHB] (123.7000,458.9000) -- (120.4000,458.9000) -- (120.4000,460.8000) -- (123.7000,460.8000) -- (123.7000,462.7000) -- (120.4000,462.7000) -- (120.4000,467.5000) -- (118.4000,467.5000) -- (118.4000,456.9000) -- (123.7000,456.9000) -- cycle;
	%-------------------------------------------------------------------------------------------
	% Applied
	\path[fill=FHB] (32.6000,58.9000) -- (34.7000,58.9000) -- (38.8000,69.5000) -- (36.6000,69.5000) -- (35.8000,67.3000) -- (31.5000,67.3000) -- (30.6000,69.5000) -- (28.5000,69.5000) -- cycle(33.6000,61.7000) -- (32.2000,65.4000) -- (35.0000,65.4000) -- cycle;
	% aPplied
	\path[fill=FHB] (41.4000,58.9000) -- (43.5000,58.9000) .. controls (44.7000,58.9000) and (45.5000,59.0000) .. (46.0000,59.2000) .. controls (46.5000,59.4000) and (46.9000,59.8000) .. (47.2000,60.2000) .. controls (47.5000,60.7000) and (47.6000,61.3000) .. (47.6000,61.9000) .. controls (47.6000,62.6000) and (47.4000,63.3000) .. (47.0000,63.7000) .. controls (46.6000,64.2000) and (46.1000,64.5000) .. (45.4000,64.7000) .. controls (45.0000,64.8000) and (44.3000,64.9000) .. (43.3000,64.9000) -- (43.3000,69.4000) -- (41.3000,69.4000) -- (41.3000,58.9000) -- cycle(43.5000,63.0000) -- (44.1000,63.0000) .. controls (44.6000,63.0000) and (45.0000,63.0000) .. (45.2000,62.9000) .. controls (45.4000,62.8000) and (45.6000,62.7000) .. (45.7000,62.5000) .. controls (45.8000,62.3000) and (45.9000,62.1000) .. (45.9000,61.9000) .. controls (45.9000,61.5000) and (45.7000,61.2000) .. (45.4000,61.0000) .. controls (45.2000,60.9000) and (44.7000,60.8000) .. (44.1000,60.8000) -- (43.5000,60.8000) -- cycle;
	% apPlied
	\path[fill=FHB] (50.9000,58.9000) -- (53.0000,58.9000) .. controls (54.2000,58.9000) and (55.0000,59.0000) .. (55.5000,59.2000) .. controls (56.0000,59.4000) and (56.4000,59.8000) .. (56.7000,60.2000) .. controls (57.0000,60.7000) and (57.1000,61.3000) .. (57.1000,61.9000) .. controls (57.1000,62.6000) and (56.9000,63.3000) .. (56.5000,63.7000) .. controls (56.1000,64.2000) and (55.6000,64.5000) .. (54.9000,64.7000) .. controls (54.5000,64.8000) and (53.8000,64.9000) .. (52.8000,64.9000) -- (52.8000,69.4000) -- (50.8000,69.4000) -- (50.8000,58.9000) -- cycle(52.9000,63.0000) -- (53.5000,63.0000) .. controls (54.0000,63.0000) and (54.4000,63.0000) .. (54.6000,62.9000) .. controls (54.8000,62.8000) and (55.0000,62.7000) .. (55.1000,62.5000) .. controls (55.2000,62.3000) and (55.3000,62.1000) .. (55.3000,61.9000) .. controls (55.3000,61.5000) and (55.1000,61.2000) .. (54.8000,61.0000) .. controls (54.6000,60.9000) and (54.1000,60.8000) .. (53.5000,60.8000) -- (52.9000,60.8000) -- cycle;
	% appLied
	\path[shift={(-102.4,-398.0)},fill=FHB] (164.8000,465.6000) -- (167.7000,465.6000) -- (167.7000,467.5000) -- (162.8000,467.5000) -- (162.8000,456.9000) -- (164.8000,456.9000) -- cycle;
	% applIed
	\path[fill=FHB] (67.9000,58.9000) rectangle (69.9000,69.5000);
	% appliEd
	\path[shift={(-102.4,-398.0)},fill=FHB] (181.6000,458.9000) -- (177.8000,458.9000) -- (177.8000,460.8000) -- (181.6000,460.8000) -- (181.6000,462.7000) -- (177.8000,462.7000) -- (177.8000,465.5000) -- (181.6000,465.5000) -- (181.6000,467.5000) -- (175.8000,467.5000) -- (175.8000,456.9000) -- (181.6000,456.9000) -- cycle;
	% applieD
	\path[fill=FHB] (82.3000,58.9000) -- (84.7000,58.9000) .. controls (86.2000,58.9000) and (87.4000,59.1000) .. (88.1000,59.5000) .. controls (88.9000,59.9000) and (89.5000,60.5000) .. (90.0000,61.4000) .. controls (90.5000,62.3000) and (90.7000,63.3000) .. (90.7000,64.4000) .. controls (90.7000,65.2000) and (90.6000,66.0000) .. (90.3000,66.7000) .. controls (90.0000,67.4000) and (89.7000,68.0000) .. (89.2000,68.4000) .. controls (88.7000,68.9000) and (88.2000,69.2000) .. (87.6000,69.3000) .. controls (87.0000,69.5000) and (86.1000,69.6000) .. (84.7000,69.6000) -- (82.3000,69.6000) -- cycle(84.3000,60.8000) -- (84.3000,67.5000) -- (85.2000,67.5000) .. controls (86.1000,67.5000) and (86.8000,67.4000) .. (87.2000,67.2000) .. controls (87.6000,67.0000) and (88.0000,66.6000) .. (88.2000,66.1000) .. controls (88.5000,65.6000) and (88.6000,65.0000) .. (88.6000,64.2000) .. controls (88.6000,63.1000) and (88.3000,62.2000) .. (87.7000,61.6000) .. controls (87.1000,61.0000) and (86.2000,60.8000) .. (85.0000,60.8000) -- cycle;
	%-------------------------------------------------------------------------------------------
	% Sciences
	\path[fill=FHB] (6.5000,78.1000) -- (5.0000,79.4000) .. controls (4.5000,78.7000) and (3.9000,78.3000) .. (3.4000,78.3000) .. controls (3.1000,78.3000) and (2.9000,78.4000) .. (2.7000,78.5000) .. controls (2.5000,78.6000) and (2.4000,78.8000) .. (2.4000,79.0000) .. controls (2.4000,79.2000) and (2.5000,79.3000) .. (2.6000,79.5000) .. controls (2.8000,79.7000) and (3.3000,80.2000) .. (4.1000,80.9000) .. controls (4.9000,81.5000) and (5.3000,82.0000) .. (5.5000,82.1000) .. controls (5.9000,82.5000) and (6.2000,82.9000) .. (6.4000,83.3000) .. controls (6.6000,83.7000) and (6.7000,84.1000) .. (6.7000,84.5000) .. controls (6.7000,85.4000) and (6.4000,86.1000) .. (5.8000,86.7000) .. controls (5.2000,87.3000) and (4.4000,87.6000) .. (3.4000,87.6000) .. controls (2.6000,87.6000) and (2.0000,87.4000) .. (1.4000,87.0000) .. controls (0.8000,86.6000) and (0.4000,86.0000) .. (0.0000,85.3000) -- (1.7000,84.3000) .. controls (2.2000,85.2000) and (2.8000,85.7000) .. (3.5000,85.7000) .. controls (3.8000,85.7000) and (4.1000,85.6000) .. (4.4000,85.4000) .. controls (4.6000,85.2000) and (4.8000,85.0000) .. (4.8000,84.7000) .. controls (4.8000,84.5000) and (4.7000,84.2000) .. (4.5000,84.0000) .. controls (4.3000,83.8000) and (3.9000,83.4000) .. (3.3000,82.9000) .. controls (2.1000,82.0000) and (1.4000,81.2000) .. (1.1000,80.7000) .. controls (0.8000,80.2000) and (0.6000,79.7000) .. (0.6000,79.2000) .. controls (0.6000,78.5000) and (0.9000,77.8000) .. (1.4000,77.3000) .. controls (2.0000,76.8000) and (2.7000,76.5000) .. (3.5000,76.5000) .. controls (4.0000,76.5000) and (4.5000,76.6000) .. (5.0000,76.9000) .. controls (5.4000,77.0000) and (5.9000,77.4000) .. (6.5000,78.1000);
	% sCiences
	\path[fill=FHB] (21.6000,78.5000) -- (20.2000,79.9000) .. controls (19.2000,78.9000) and (18.2000,78.4000) .. (17.0000,78.4000) .. controls (16.0000,78.4000) and (15.1000,78.7000) .. (14.4000,79.4000) .. controls (13.7000,80.1000) and (13.4000,80.9000) .. (13.4000,82.0000) .. controls (13.4000,82.7000) and (13.6000,83.3000) .. (13.9000,83.9000) .. controls (14.2000,84.4000) and (14.6000,84.9000) .. (15.2000,85.2000) .. controls (15.8000,85.5000) and (16.4000,85.7000) .. (17.1000,85.7000) .. controls (17.7000,85.7000) and (18.2000,85.6000) .. (18.7000,85.4000) .. controls (19.2000,85.2000) and (19.7000,84.8000) .. (20.3000,84.2000) -- (21.7000,85.6000) .. controls (20.9000,86.4000) and (20.2000,86.9000) .. (19.5000,87.2000) .. controls (18.8000,87.5000) and (18.0000,87.6000) .. (17.1000,87.6000) .. controls (15.4000,87.6000) and (14.1000,87.1000) .. (13.0000,86.0000) .. controls (11.9000,85.0000) and (11.4000,83.6000) .. (11.4000,82.0000) .. controls (11.4000,80.9000) and (11.6000,80.0000) .. (12.1000,79.2000) .. controls (12.6000,78.4000) and (13.3000,77.7000) .. (14.2000,77.2000) .. controls (15.1000,76.7000) and (16.1000,76.4000) .. (17.1000,76.4000) .. controls (18.0000,76.4000) and (18.8000,76.6000) .. (19.6000,77.0000) .. controls (20.3000,77.3000) and (21.0000,77.8000) .. (21.6000,78.5000);
	% scIences
	\path[fill=FHB] (26.5000,76.6000) rectangle (28.5000,87.2000);
	% sciEnces
	\path[shift={(-102.4,-398.0)},fill=FHB] (142.3000,476.6000) -- (138.5000,476.6000) -- (138.5000,478.5000) -- (142.3000,478.5000) -- (142.3000,480.5000) -- (138.5000,480.5000) -- (138.5000,483.3000) -- (142.3000,483.3000) -- (142.3000,485.3000) -- (136.5000,485.3000) -- (136.5000,474.6000) -- (142.3000,474.6000) -- cycle;
	% scieNces
	\path[shift={(-102.4,-398.0)},fill=FHB] (153.9000,481.6000) -- (153.9000,474.6000) -- (155.9000,474.6000) -- (155.9000,485.3000) -- (154.0000,485.3000) -- (149.4000,478.3000) -- (149.4000,485.3000) -- (147.4000,485.3000) -- (147.4000,474.6000) -- (149.3000,474.6000) -- cycle;
	% scienCes
	\path[fill=FHB] (69.0000,78.5000) -- (67.6000,79.9000) .. controls (66.6000,78.9000) and (65.6000,78.4000) .. (64.4000,78.4000) .. controls (63.4000,78.4000) and (62.5000,78.7000) .. (61.8000,79.4000) .. controls (61.1000,80.1000) and (60.8000,80.9000) .. (60.8000,82.0000) .. controls (60.8000,82.7000) and (61.0000,83.3000) .. (61.3000,83.9000) .. controls (61.6000,84.4000) and (62.0000,84.9000) .. (62.6000,85.2000) .. controls (63.2000,85.5000) and (63.8000,85.7000) .. (64.5000,85.7000) .. controls (65.1000,85.7000) and (65.6000,85.6000) .. (66.1000,85.4000) .. controls (66.6000,85.2000) and (67.1000,84.8000) .. (67.7000,84.2000) -- (69.1000,85.6000) .. controls (68.3000,86.4000) and (67.6000,86.9000) .. (66.9000,87.2000) .. controls (66.2000,87.5000) and (65.4000,87.6000) .. (64.5000,87.6000) .. controls (62.8000,87.6000) and (61.5000,87.1000) .. (60.4000,86.0000) .. controls (59.3000,85.0000) and (58.8000,83.6000) .. (58.8000,82.0000) .. controls (58.8000,80.9000) and (59.0000,80.0000) .. (59.5000,79.2000) .. controls (60.0000,78.4000) and (60.7000,77.7000) .. (61.6000,77.2000) .. controls (62.5000,76.7000) and (63.5000,76.4000) .. (64.5000,76.4000) .. controls (65.4000,76.4000) and (66.2000,76.6000) .. (67.0000,77.0000) .. controls (67.7000,77.3000) and (68.4000,77.8000) .. (69.0000,78.5000);
	% sciencEs
	\path[shift={(-102.4,-398.0)},fill=FHB] (182.2000,476.6000) -- (178.4000,476.6000) -- (178.4000,478.5000) -- (182.2000,478.5000) -- (182.2000,480.5000) -- (178.4000,480.5000) -- (178.4000,483.3000) -- (182.2000,483.3000) -- (182.2000,485.3000) -- (176.4000,485.3000) -- (176.4000,474.6000) -- (182.2000,474.6000) -- cycle;
	% scienceS
	\path[fill=FHB] (90.5000,78.1000) -- (89.0000,79.4000) .. controls (88.5000,78.7000) and (87.9000,78.3000) .. (87.4000,78.3000) .. controls (87.1000,78.3000) and (86.9000,78.4000) .. (86.7000,78.5000) .. controls (86.5000,78.6000) and (86.4000,78.8000) .. (86.4000,79.0000) .. controls (86.4000,79.2000) and (86.5000,79.3000) .. (86.6000,79.5000) .. controls (86.8000,79.7000) and (87.3000,80.2000) .. (88.1000,80.9000) .. controls (88.9000,81.5000) and (89.3000,82.0000) .. (89.5000,82.1000) .. controls (89.9000,82.5000) and (90.2000,82.9000) .. (90.4000,83.3000) .. controls (90.6000,83.7000) and (90.7000,84.1000) .. (90.7000,84.5000) .. controls (90.7000,85.4000) and (90.4000,86.1000) .. (89.8000,86.7000) .. controls (89.2000,87.3000) and (88.4000,87.6000) .. (87.4000,87.6000) .. controls (86.6000,87.6000) and (86.0000,87.4000) .. (85.4000,87.0000) .. controls (84.8000,86.6000) and (84.4000,86.0000) .. (84.0000,85.3000) -- (85.7000,84.3000) .. controls (86.2000,85.2000) and (86.8000,85.7000) .. (87.5000,85.7000) .. controls (87.8000,85.7000) and (88.1000,85.6000) .. (88.4000,85.4000) .. controls (88.6000,85.2000) and (88.8000,85.0000) .. (88.8000,84.7000) .. controls (88.8000,84.5000) and (88.7000,84.2000) .. (88.5000,84.0000) .. controls (88.3000,83.8000) and (87.9000,83.4000) .. (87.3000,82.9000) .. controls (86.1000,82.0000) and (85.4000,81.2000) .. (85.1000,80.7000) .. controls (84.8000,80.2000) and (84.6000,79.7000) .. (84.6000,79.2000) .. controls (84.6000,78.5000) and (84.9000,77.8000) .. (85.4000,77.3000) .. controls (86.0000,76.8000) and (86.7000,76.5000) .. (87.5000,76.5000) .. controls (88.0000,76.5000) and (88.5000,76.6000) .. (89.0000,76.9000) .. controls (89.4000,77.0000) and (89.9000,77.4000) .. (90.5000,78.1000);
	%-------------------------------------------------------------------------------------------
	% Linie
	\path[fill=FHB] (104.7000,101.2000) -- (104.7000,5.2000) .. controls (104.7000,4.7000) and (105.1000,4.3000) .. (105.6000,4.3000) .. controls (106.1000,4.3000) and (106.5000,4.7000) .. (106.5000,5.2000) -- (106.5000,101.2000) .. controls (106.5000,101.7000) and (106.1000,102.1000) .. (105.6000,102.1000) .. controls (105.1000,102.1000) and (104.7000,101.7000) .. (104.7000,101.2000);
	\end{tikzpicture}
}
\end{minipage}
\vspace{0.1cm}\\	
\raggedright
\begin{minipage}{0.65\textwidth}
\sffamily
\raggedright
\footnotesize
+43 5 90500-5110\\
\href{mailto:info@fh-kaernten.at}{info@fh-kaernten.at}\\
\href{http://www.fh-kaernten.at}{www.fh-kaernten.at}\\
\end{minipage}

\vspace{2,95cm}
%\definecolor{titel}{cmyk}{.68,.39,0,0}
%\color{titel} 

\centering
	\sodef\spaceout{\bfseries\Huge\centering}{0pt plus 1fil}{.4em plus 1fil}{0pt}
	\makebox[15,65cm][l]{\spaceout{PROJEKTARBEIT}} \\
%\color{black} 	
	\vspace{0,53cm}
	{\huge\bfseries \Projekttitel\\}
	\vspace{0,9cm}
	{\large \Projektuntertitel\\}
	\vspace{0,6cm}
	{\large im Studiengang\\}
	{\large\bfseries \Studiengang \\}
	{\large Vertiefung: \Vertiefungsrichtung\\}
	\vspace{0,53cm}
	%{\large Zur Erlangung des akademischen Grades\\}
	%{\large DI für technisch-wissenschaftliche Berufe\\}
%	\vspace{0,4cm}
	\vspace{\fill}
	\raggedright
\begin{minipage}{5,0cm}
	\sffamily
	\raggedright
	Verfasser: \\
	\vspace{\baselineskip}
	Matrikelnummer:\\
	\vspace{\baselineskip}
	Betreuung:\\
	\vspace{\baselineskip}
	~\\
	\vspace{\baselineskip}
	~\\
\end{minipage}
\begin{minipage}{0.45\textwidth}
	\sffamily
	\raggedright
	\Student \\
	\vspace{\baselineskip} 
	\Matrikelnummer\\
	\vspace{\baselineskip}
	\BetreuerA \\
	\vspace{\baselineskip}	
	\BetreuerB \\
	\vspace{\baselineskip}
	\BetreuerC \\
\end{minipage}
\hfill
\begin{minipage}{5,0cm}
	\sffamily
	\raggedright
	\hfill
\end{minipage}
\begin{minipage}{0.45\textwidth}
	\sffamily
	\raggedright

\end{minipage}


\vspace{26mm}
\parbox{7.7cm}{\centering\hrule\medskip Ort, Datum der Abgabe}
\vspace{0.0cm}
\hfill
\parbox{0.7cm}{ }
\parbox{6cm}{\raggedleft\medskip Stempel des Studiengangs}
\vspace{0.0cm}
\end{titlepage}
\pagebreak[4]
\restoregeometry
\clearscrheadings% Bereinigung Kopfzeile
\ifoot[\Student]{\parbox[b][5mm]{5cm}{\Student}}
\ofoot[\pagemark]{\parbox[b][5mm]{5cm}{\raggedleft \pagemark}}
\renewcommand{\arbeit}{PROJEKTARBEIT }% Titelseite für die Projektarbeit
%\input{document/titelseite_da}% Titelseite für die Diplomarbeit
%***********************************************************


\vspace{2cm}
\begin{flushleft}
\begin{tabularx}{\textwidth}[t]{@{}lll@{}}
Name: &\hspace*{2mm}& \Student \\
Matrikelnummer: &\hspace*{2mm}& \Matrikelnummer \\
Geburtstag: &\hspace*{2mm}& \Geburtstag \\
Adresse: &\hspace*{2mm}& \Strasse \\
&\hspace*{2mm}& \Wohnort 
\end{tabularx}
\end{flushleft}
\vspace*{20mm}

{\Large Eidesstattliche Erklärung}\\
\ohead[]{\parbox[t][5mm]{\textwidth}{\raggedleft Eidesstattliche Erklärung}}


Ich erkläre hiermit,\\
\begin{itemize}
\item dass ich die vorliegende \arbeit selbstständig und ohne fremde Hilfe verfasst und noch nicht anderweitig zu Prüfungszwecken vorgelegt habe.
\item dass ich keine anderen als die angegebenen Hilfsmittel benutzt, die den verwendeten Quellen wörtlich oder inhaltlich entnommenen Stellen als solche kenntlich gemacht und mich sonst keiner unerlaubten Hilfe bedient habe.
\item dass die elektronisch abgegebene Fassung mit der in Papierform eingereichten Arbeit übereinstimmt.
\item dass ich einwillige, dass ein Belegexemplar der von mir erstellten \arbeit in den Bestand der Fachhochschulbibliothek aufgenommen und benutzbar gemacht wird (Veröffentlichung gemäß §8 UrhG).
\end{itemize} 

\vfill
{\parbox{5cm}{\centering\hrule\medskip Ort, Datum}\hfill \parbox{\widthof{Unterschrift \Student}}{\centering\hrule\medskip Unterschrift \Student}}
\vspace{1cm}
\null
\clearpage


% TODO: Anpassung des Inhalts oder auskommentieren
\section*{Sperrvermerk}
\ohead[Sperrvermerk]{Sperrvermerk}

Die vorliegende \arbeit mit dem Titel:\medskip

\hfill\parbox{0.95\textwidth}{%
	\textbf{\Projekttitel}\\
	\small\Projektuntertitel}\medskip

beinhaltet interne und vertrauliche Informationen.\medskip

\textbf{Die Einsicht in diese \arbeit ist nicht gestattet.}\medskip

Ausgenommen davon sind jene Personen, welche die Arbeit betreuten sowie befugte Mitglieder des Prüfungsausschusses. 
Veröffentlichung und/oder Vervielfältigung der \arbeit --- auch in Auszügen --- sind ebenfalls nicht gestattet.\medskip

Ausnahmen dieser Regelung bedürfen einer Genehmigung durch:\medskip

% TODO: Angabe von Person/Unternehmen/etc.******************
Fritz Musterknabe
%***********************************************************
\clearpage
% kann auskommentiert werden.
%***********************************************************

\null\vfill
\ohead[]{\parbox[t][5mm]{\textwidth}{\raggedleft Gender-Erklärung}}
\section*{Gender-Erklärung}
Aus Gründen der besseren Lesbarkeit wird in dieser Diplomarbeit  die Sprachform des generischen Maskulinums angewendet. Es wird an dieser Stelle darauf hingewiesen, dass die ausschließliche Verwendung der männlichen Form geschlechtsunabhängig verstanden werden soll.   
\vfill\null
\clearpage% kann auskommentiert werden.

% TODO: Anpassung des Inhalts, Danksagung kann auch auskommentiert werden
\null\vfill
\ohead[]{\parbox[t][5mm]{\textwidth}{\raggedleft Danksagung}}
\section*{Danksagung}

\lipsum[1-2]
\vfill\null
\clearpage% kann auskommentiert werden.
\section*{Kurzfassung}\ohead[]{\parbox[t][5mm]{\textwidth}{\raggedleft Kurzfassung}}
\lipsum[1-2]

\keywordsDE{Ipsum, Lorem, Worte}

\clearpage

\section*{Abstract}\ohead[]{\parbox[t][5mm]{\textwidth}{\raggedleft Abstract}}
\begin{otherlanguage}{english}
\lipsum[1-2]
	
	\keywordsEN{Words, don't, come, easy}
\end{otherlanguage}
\clearpage
{\large Abkürzungsverzeichnis}\\

\ohead[]{\parbox[t][5mm]{\textwidth}{\raggedleft Abkürzungsverzeichnis}}
{\small \begin{tabular}[t]{ll}
	$\coloneqq$ & \textit{definiert als}, bei Neudefinitionen mit ursprünglichem Wert in der Gleichung\\
	$P_{i}$ & bekannter Punkt auf der Trajektorie, gewählt (Startpunkt) oder ermittelt (Iteration)\\
	$P_{i+1}$ & nächster, zu ermittelnder Punkt der Trajektorie\\
	$\varphi$ & Winkel zwischen der x-Achse und der Geraden $\overline{P_{i}P_{i+1}}$\\
	$\varphi_{i}$ & Hauptspannungswinkel am Punkt $P_{i}$ \\
	$\varphi_{i+1}$ & Hauptspannungswinkel am Punkt $P_{i+1}$ \\
	$r$ & Iterationsschritt, Abstand der Punkte $P_{i}$ und $P_{i+1}$\\
	$S$ & Schnittpunkt der Verlängerung der Hauptspannungswinkel $\varphi_{i}$ und $\varphi_{i+1}$\\
	$g_{i}$ & Gerade zwischen Punkt $P_{i}$ und dem Schnittpunkt $S$ \\
	$g_{i+1}$ & Gerade zwischen Punkt $P_{i+1}$ und dem Schnittpunkt $S$ \\
	$l_{i}$ & Länge der Geraden $g_{i}$\\
	$l_{i+1}$ & Länge der Geraden $g_{i+1}$\\
\end{tabular}}
\clearpage
% kann auskommentiert werden.
%***********************************************************
% TODO: Anpassung der längsten Inhaltsnummerierung für Abstand
\settowidth{\tocindent}{0.0.0}% Berechnung Breite der Nummerierung
%***********************************************************
\inhaltsverzeichnis

% TODO: Erstellung von Text in Untergeordneten Dateine (am besten Kapitelweise)
\section{Einleitung}
\label{sec:einleitung}
\textbf{Problemstellung}\vspace{1mm}\\
\lipsum[1]

\textbf{Ziel}\vspace{1mm}\\
\lipsum[1-2]

\clearpage
\textbf{Methode}\vspace{1mm}\\
\lipsum[1]

\clearpage
\section{Was ist LaTeX und wofür ist es gut?
	\label{sec:latex}}
{\LaTeX} (sprich "`Latech"') ist ein Softwarepaket, das die Benutzung des Textsatzsystems TeX mit Hilfe von Makros vereinfacht. LaTeX liegt derzeit in der Version $2_\varepsilon$ vor.

Im Gegensatz zu anderen Textverarbeitungsprogrammen, die nach dem What-you-see-is-what-you-get-Prinzip funktionieren, arbeitet der Autor mit Textdateien, in denen er innerhalb eines Textes anders zu formatierende Passagen oder Überschriften mit Befehlen textuell auszeichnet. Das Beispiel unten zeigt den Quellcode eines einfachen LaTeX-Dokuments. Bevor das LaTeX-System den Text entsprechend setzen kann, muss es den Quellcode verarbeiten. Das dabei von LaTeX generierte Layout gilt als sehr sauber, sein Formelsatz als sehr ausgereift. Außerdem ist die Ausgabe u. a. nach PDF, HTML und PostScript möglich. LaTeX eignet sich insbesondere für umfangreiche Arbeiten wie Diplomarbeiten und Dissertationen, die oftmals strengen typographischen Ansprüchen genügen müssen. Insbesondere in der Mathematik und den Naturwissenschaften erleichtert LaTeX das Anfertigen von Schriftstücken durch seine komfortablen Möglichkeiten der Formelsetzung gegenüber üblichen Textverarbeitungssystemen. Das Verfahren von LaTeX wird auch mit WYSIWYAF (What you see is what you asked for.) umschrieben.

Das schrittweise Arbeiten erfordert vordergründig im Vergleich zu herkömmlichen Textverarbeitungen einerseits eine längere Einarbeitungszeit, andererseits kann das Aussehen des Resultats genau festgelegt werden. Die längere Einarbeitungszeit kann sich jedoch, insbesondere bei Folgeprojekten mit vergleichbarem Umfang oder ähnlichen Erfordernissen, lohnen.[8] Inzwischen gibt es auch grafische Editoren, die mit LaTeX arbeiten können und WYSIWYG oder WYSIWYM (What you see is what you mean.) bieten und ungeübten Usern den Einstieg deutlich erleichtern können. Beispiele für LaTeX-Entwicklungsumgebungen sind im Abschnitt Entwicklungsumgebungen aufgelistet.\footnote{Quelle: \href{https://de.wikipedia.org/wiki/LaTeX}{wikipedia.org}}

\clearpage

\subsection{Was brauche ich für Programme und was kostet der Spaß?
	\label{sec:latex_programme}}
Zum einen benötigt man eine TeX-Distribution, zum anderen einen Editor. hier reicht theoretisch schon der in Windows verbaute \textbf{notepad}. Es gibt allerdings Editoren, die speziell auf {\LaTeX} ausgerichtet sind. Zur Quellenverwaltung kann Citavi oder die kostenlose Alternative "`JabRef"' Verwendung finden.

\begin{center}
	\begin{tabular}{ccc}
	\href{https://miktex.org/}{Tex-Distribution MiKTeX} &
	\href{https://www.texstudio.org/}{Editor TeXstudio} &
	\href{http://www.jabref.org/}{JabRef} \\
\end{tabular}
\end{center}

Bei der Installation kann man eigentlich kaum etwas falsch machen. Bei MiKTeX empfiehlt es sich die Option "`Install packages on the fly"' zu aktivieren. So erspart man sich das manuelle Laden von Paketen. 

{\LaTeX} wie auch die meiste benötigte Software sind Open Source, das heißt, dass hier keine Kosten anfallen.

\subsection{Wo finde ich Hilfe wenn ich etwas brauche? 
	\label{sec:latex_hilfe}}
\begin{itemize}
	\item \href{http://projekte.dante.de/DanteFAQ/WebHome}{Projekt dante - \LaTeX -FAQ}
	\item \href{https://tex.stackexchange.com/}{TeX-Stackexchange}
\end{itemize}
\textcolor{red}{Am besten nun noch die Starteinstellungen "`Startprofil.txsprofile"' über die Optionen laden.}
\clearpage

\section{Kapitel der ersten Ebene}
Kapitelüberschriften werden mit \texttt{\textbackslash section\{Titel\}} erzeugt. Die Unterkapitel mit der Erweiterung des Befehls \texttt{sub} zu \texttt{\textbackslash subsection\{Titel\}} bzw. \texttt{\textbackslash subsubsection\{Titel\}} eine tiefere Gliederung erfordert eine manuelle Erstellung des entsprechenden Befehls und ist als nicht sinnvoll zu erachten, da Dokumente sonst unübersichtlich werden. Es gibt hier allerdings noch die Möglichkeit des Paragraphen. Die Nummerierung der Kapitel erfolgt automatisch.
\subsection{Kapitel der zweiten Ebene}
Der Text beginnt ganz normal hier. Die Nummerierung des Kapitels erfolgt automatisch.
\subsubsection{Kapitel der dritten Ebene}
Der Text beginnt ganz normal hier. Die Nummerierung des Kapitels erfolgt automatisch.
\paragraph{Paragraph}
Ein Paragraph erscheint nicht im Inhaltsverzeichnis. der Text wird an den Titel des Paragraphen angehängt.

\section*{Kapitel der ersten Ebene (nicht im Inhaltsverzeichnis)}
Der Text beginnt ganz normal hier. Die Nummerierung des Kapitels erfolgt \textbf{nicht} und sie erscheinen weder im Inhaltsverzeichnis noch in der Kopfzeile. In diesem Fall ist allerdings manuell eine Eintragung ins Inhaltsverzeichnis erfolgt. 

Diese Eintragung mit \texttt{\textbackslash addcontentsline\{toc\}\{section\}\{Ein Kapitel ohne Nummerierung\}} ist im Fließtext nicht sichtbar.
\addcontentsline{toc}{section}{Ein Kapitel ohne Nummerierung}

Der hier verwendete Code \texttt{\textbackslash section\textbf{*}\{Titel\}} ist analog auch für \texttt{subsection} und \texttt{subsubsection} anwendbar
\clearpage
\section{Mathematische Formel}
\label{sec:mathematischeformel}
Nachfolgend sehen wir uns verschiedene Formeln an
\subsection{Eigenständige Formel mit Erklärung}
\label{sec:eq_mit_erklaerung}
Die Erklärung beschreibt die in der Formel verwendeten Zeichen
\begin{equation}
\delta = \frac{C}{\lambda}\left(\sigma_{1}-\sigma_{2}\right)d
\label{eq:HG_Spannungsoptik}
\end{equation}
\begin{conditions}
	\sigma_{1,2} & Hauptspannungen [$kN/mm^{2}$]\\
	d & Modellstärke [$mm$]\\
	\delta & Phasenverschiebung [$-$]\\
	C & Materialkonstante [$mm^{2}/kN$]\\
	\lambda & Wellenlänge des Lichts im Vakuum [$nm$]
\end{conditions}

\subsection{Formelbündelung}
\label{sec:formelbuendel}
Formeln können als Subequations gebündelt werden. Die Umgebung "`align"' hilft bei der Ausrichtung
\begin{subequations} \label{grp:phi}
	\begin{align}
	\varphi_{\text{\tiny I}} &= \varphi_{i+1} \label{eq:phi_1}\\
	\varphi_{\text{\tiny II}} &= \varphi_{i+1}+\frac{1}{2}\pi \label{eq:phi_2}\\
	\varphi_{\text{\tiny III}} &= \varphi_{i+1}+\pi \label{eq:phi_3}\\
	\varphi_{\text{\tiny  IV}} &= \varphi_{i+1}+\frac{3}{2}\pi \label{eq:phi_4}
	\end{align}
\end{subequations}

\subsection{Mathematikmodus im Text}
\label{sec:textmathe}
Man kann auch den Mathematikmodus im Text selbst einschalten, wie nachfolgend zu sehen: $\pi \approx 3.141593$. So kann man unter anderem auch auf griechische Buchstaben verweisen.
\clearpage
\section{Bilder}
\label{sec:bilder}
Normalerweise sind Bilder und Tabellen "`float"' Objekte, das heißt, dass Latex diese Objekte an den bestmöglichen Ort verschiebt. 
\begin{figure}[H]
	\centering
	\includegraphics[height=5cm]{example-image-b} 
	\caption[Dies ist eine kürzere Bildunterschrift (Verzeichnis)]{Dies ist eine längere Bildunterschrift welche unter dem Bild angezeigt wird. Ist keine kurze Unterschrift angegeben so wird diese lange Bildunterschrift im Verzeichnis angezeigt.}
	\label{fig:testbild}
\end{figure}
Will man dies nicht, so kann man mit der Option [H] (hier) das Bild an genau die Stelle im Text platzieren wo man es haben möchte. das bedingt natürlich dass der Platz unter Umständen nicht optimal genutzt wird.
\begin{figure}[H]
	\centering
	\includegraphics[width=0.7\linewidth]{bilder/newton}
	\caption{Newton entdeckt die Gravitation}
	\label{fig:newton}
\end{figure}

Mit Latex bzw mit dem Paket TiKz und PGF lässt sich sogar zeichnen. Näheres dazu hier: \href{http://www.texample.net/tikz/examples/}{texample.net}. Es gibt auch die Möglichkeit Graphen und Plots in Latex zu erstellen:
\begin{figure}[H]
	\centering
	\begin{tikzpicture}
		\begin{axis}[
			axis lines = left,
			xlabel = $x$,
			ylabel = {$f(x)$},
			]
			%Below the red parabola is defined
			\addplot [
			domain=-10:10, 
			samples=100, 
			color=red,
			]
			{x^2 - 2*x - 1};
			\addlegendentry{$x^2 - 2x - 1$}
			%Here the blue parabloa is defined
			\addplot [
			domain=-10:10, 
			samples=100, 
			color=blue,
			]
			{x^2 + 2*x + 1};
			\addlegendentry{$x^2 + 2x + 1$}
		\end{axis}
	\end{tikzpicture}
	\caption{Plot der Funktionen $x^2 - 2x - 1$ und $x^2 + 2x + 1$}
\end{figure}

\begin{figure}[H]
	\centering
\begin{minipage}[c]{0.45\linewidth}
	\begin{tikzpicture}
	\begin{axis}[
	enlargelimits=true,
	legend style
	={
		cells={anchor=east},
		legend pos
		=outer north east,
	},
	]
	\addplot[color=red,only marks,mark size=2.9pt]
	table[x index=0,y index=1] 
	{data/scattered_example.dat};
	\addlegendentry{ds1}
	\addplot[color=blue,only marks,mark=square*,mark size=2.9pt]
	table[x index=0,y index=2] 
	{data/scattered_example.dat};
	\addlegendentry{ds2}
		\addplot[color=green,only marks,mark=triangle*,mark size=2.9pt]
	table[x index=0,y index=3] 
	{data/scattered_example.dat};
	\addlegendentry{ds3}
		\addplot[color=black,only marks,mark=x,mark size=2.9pt]
	table[x index=0,y index=4] 
	{data/scattered_example.dat};
	\addlegendentry{ds4}
	\end{axis}
	\end{tikzpicture}
\end{minipage}
\begin{minipage}[c]{0.45\linewidth}\hspace{1.2cm}
	\begin{tabular}{lllll}
		base&ds1&ds2&ds3&ds4\\ \toprule
		1&5,4&1&1,8&7,2\\
		2&1,8&0,2&8,1&8\\
		3&2,4&0,2&0,9&4,8\\
		4&3,6&1,6&8,1&5,6\\
		5&0,6&1&1,8&6,4\\
		6&4,8&1,4&1,8&2,4\\
		7&5,4&0,2&7,2&3,2\\
		8&6&0,8&0,9&1,6\\
		9&2,4&1&7,2&1,6\\
		10&3,6&0,6&4,5&7,2\\
		11&4,8&1&1,8&3,2\\
		12&3&2&4,5&5,6\\
		13&3&0,4&7,2&5,6\\
		14&4,2&1,8&0,9&8\\
		15&3&2&5,4&5,6
	\end{tabular}
\end{minipage}
\caption{Scatterplot der Datentabelle im bezug zur Basis "`base"'}
\end{figure}
\begin{figure}[H]
	\centering
	\begin{tikzpicture}[baseline, scale=0.6]
	\pie[rotate=90, /tikz/nodes={text opacity=0,overlay}, color={yellow!40, blue!30}]{2/a, 98/b};
	\end{tikzpicture}
	\begin{tikzpicture}[baseline]
	\draw [fill=yellow!40](0,1.0) rectangle (0.3,1.3);
	\node [text width=5cm,align=left, anchor=west] at (0.5,1.15) {an einem veregneten Sommerabend auf dem Zeltplatz zwischen Mücken und Matsch};
	\draw [fill=blue!30](0,-1.0) rectangle (0.3,-1.3);
	\node [text width=5cm,align=left, anchor=west] at (0.5,-1.15) {an einem verregneten Herbstabend beim Gedanken an den Sommer};
	\end{tikzpicture}
	\caption{Wann ein Zelt-Wochenende eine tolle Idee ist}
\end{figure}
\clearpage
\section{Aufzählungen}
\label{sec:auflistung}
Nachfolgend ein Beispiel für eine Auflistung.
\subsection{Ansätze für eine gute wissenschaftliche Gestaltung (Links)}
\label{sec:liste}
\begin{itemize}\small
	\item \href{https://timfrey.files.wordpress.com/2008/04/graph.pdf}{Grundlagen der Darstellung: Grafiken, Tabellen und Literaturverweise (*.pdf)}
	\item \href{http://home.uni-leipzig.de/schreibportal/layout-druck-2/}{Schreibportal der Uni Leipzig}
	\begin{itemize}\footnotesize
		\item \href{http://home.uni-leipzig.de/schreibportal/typographie/}{Typographie }\hspace{0.5cm}\href{http://home.uni-leipzig.de/schreibportal/satzspiegel/}{Satzspiegel }\hspace{0.5cm}\href{http://home.uni-leipzig.de/schreibportal/tabellen-grafiken/}{ Tabellen und Grafiken }\hspace{0.5cm}\href{http://home.uni-leipzig.de/schreibportal/druck/}{Druck}
	\end{itemize}
	\item \href{https://latex.tugraz.at/}{Seite des Latex-Teams der TU-Graz}
	\item \href{https://www.youtube.com/playlist?list=PLwlC-XZXtzhg4fQiZQAsXIMSRW-iZtnTQ}{Videotutorials zu Latex der TU-Wien auf Youtube}
\end{itemize}
\clearpage
\section{Tabellen}
\label{sec:tabelle}
Tabellen sind nicht die Stärke von Latex, da sie als Code gebaut werden müssen. Texstudio hat allerdings ein Tool welches einem dabei hilft. Außerdem gibt es eine Erweiterung für Excel welches Tabellen als Latexcode exportiert. Hier ist allerdings Vorsicht angesagt.
\begin{table}[h]
	\centering
	\caption{Fälle dynamische Beanspruchungen (Erregungen) von Tragwerken}	
	\begin{tabular}{ll}
		\toprule
		\textbf{Erregungsart} & \textbf{Beispiel} \\\midrule
		harmonische Erregung & Maschinen, mutwillige Anregungen, ...\\ 
		periodische Erregung & Maschinen, Fußgänger, ...\\ 
		stoßartige Erregung & Anprall, Explosionen, ...\\ 
		beliebige, zufällige Erregung & Erdbeben, Wind, ...\\ \bottomrule
	\end{tabular} 
\end{table}
\cleardoublepage
\section{Zusammenfassung und Ausblick}
\label{sec:zusammenfassung}

\textbf{Zusammenfassung}\vspace{1mm}\\
Hier noch ein Beispiel für eine Referenz innerhalb des Dokuments z.b. auf das Kapitel \textbf{\ref{sec:bilder}} welches den Namen \textbf{\nameref{sec:bilder}} hat. dann ein \textit{Zitat}: \cite{Beyer2015}

Übrigens, das Literaturverzeichnis muss nach jedem Abändern der Literaturdatei neu Kompiliert werden: "`Tools"'$\rightarrow$"`Bibliographie (F8)"'

\clearpage
%***********************************************************

\abbildungsverzeichnis% Abbildungsverzeichnis
\tabellenverzeichnis% Tabellenverzeichnis
\literaturverzeichnis% Literaturverzeichnis

% TODO: Anpassung des Inhalts oder auskommentieren
%TODO Eintragung der Normen. Die Liste wird automatisch Aphabetisch nach Norm sortiert. Trenner ist das Semikolon ";". Benötigt zweifache Kompilierung
\begin{filecontents*}{normen.csv}
	%Norm ; Beschreibung ; Ausgabe
	ONR 23303; ON-Regel, Prüfverfahren Beton (PVB), Nationale Anwendung der Prüfnormen für Beton und seiner Ausgangsstoffe.; 2007-10
	ÖNORM B 4710-1-66-9;Beton Teil 1: Festlegung, Herstellung, Verwendung und Konformitätsnachweis (Regeln zur Umsetzung der ÖNORM EN 206-1 für Normal- und Schwerbeton).;2007-10	
	ÖNORM EN 12390-2;Prüfung von Festbeton, Teil 2. Herstellung und Lagerung von Probekörpern für Festigkeitsprüfungen.;2009-07
	DIN EN 1993-1-1;Eurocode 3: Bemessung und Konstruktion von Stahlbauten -  Teil 1-1: Allgemeine Bemessungsregeln und Regeln für den Hochbau;2010-12
	ACI 318-14;Building Code Requirements for Structural Concrete;2014-02
\end{filecontents*}
%***********************************************************
\normenverzeichnis% kann auskommentiert werden.
%***********************************************************

\label{lastpage}% letzte Seite des Inhalts
\anhangtrenner% Trennseite für Anhang
\anhangverzeichnis% Anhangverzeichnis

% TODO: Anpassung der längsten Inhaltsnummerierung für Abstand
\settowidth{\tocindent}{A.00}% Berechnung Breite der Nummerierung
%***********************************************************
% TODO: Erstellen von Anhangkapiteln
\section{Erster Anhang}
\lipsum[1]
% This file was created by matlab2tikz.
%
%The latest updates can be retrieved from
%  http://www.mathworks.com/matlabcentral/fileexchange/22022-matlab2tikz-matlab2tikz
%where you can also make suggestions and rate matlab2tikz.
%
\begin{figure}
\centering
\begin{tikzpicture}
\begin{axis}[%
width=.8\linewidth,
height=3.566in,
at={(0.758in,0.481in)},
scale only axis,
xmin=0,
xmax=20,
xlabel style={font=\color{white!15!black}},
xlabel={$Zeit \ t\left[ s\right] $},
ymin=-5.5,
ymax=5.5,
ylabel style={font=\color{white!15!black}},
ylabel={$x(t), v(t), a(t)$},
axis background/.style={fill=white},
title={Harmonische Schwingung},
xmajorgrids,
ymajorgrids,
legend style={at={(0.97,0.03)}, anchor=south east, legend cell align=left, align=left, draw=white!15!black}
]
\addplot [color=black]
  table[row sep=crcr]{%
0	5\\
0.1	4.99375130197483\\
0.2	4.97502082639013\\
0.3	4.94385538968021\\
0.4	4.90033288920621\\
0.5	4.84456210855322\\
0.6	4.77668244562803\\
0.7	4.69686356423689\\
0.8	4.60530497001443\\
0.9	4.50223551176338\\
1	4.38791280945186\\
1.1	4.26262261029753\\
1.2	4.12667807454839\\
1.3	3.98041899274528\\
1.4	3.82421093642244\\
1.5	3.6584443443691\\
1.6	3.48353354673583\\
1.7	3.29991572942491\\
1.8	3.10804984135332\\
1.9	2.90841544731942\\
2	2.7015115293407\\
2.1	2.48785523945863\\
2.2	2.26798060712789\\
2.3	2.04243720442079\\
2.4	1.81178877238337\\
2.5	1.57661181197634\\
2.6	1.33749414312294\\
2.7	1.09503343546521\\
2.8	0.849835714501204\\
2.9	0.602513846836832\\
3	0.353686008338515\\
3.1	0.103974139015462\\
3.2	-0.145997611506444\\
3.3	-0.39560444403367\\
3.4	-0.644222471477624\\
3.5	-0.89123027824746\\
3.6	-1.13601047346544\\
3.7	-1.37795123412256\\
3.8	-1.61644783431752\\
3.9	-1.85090415675644\\
4	-2.08073418273571\\
4.1	-2.30536345688357\\
4.2	-2.52423052299929\\
4.3	-2.73678832740135\\
4.4	-2.94250558627673\\
4.5	-3.1408681136137\\
4.6	-3.33138010639912\\
4.7	-3.51356538386777\\
4.8	-3.68696857770623\\
4.9	-3.85115627023654\\
5	-4.00571807773467\\
5.1	-4.15026767617611\\
5.2	-4.28444376684474\\
5.3	-4.40791097939143\\
5.4	-4.52036071008531\\
5.5	-4.62151189316232\\
5.6	-4.71111170334329\\
5.7	-4.78893618776545\\
5.8	-4.85479082574795\\
5.9	-4.90851101499227\\
6	-4.94996248300223\\
6.1	-4.97904162269531\\
6.2	-4.9956757513664\\
6.3	-4.99982329235671\\
6.4	-4.99147387897377\\
6.5	-4.97064838040273\\
6.6	-4.93739884954432\\
6.7	-4.89180839290967\\
6.8	-4.8339909628973\\
6.9	-4.76409107297152\\
7	-4.68228343645398\\
7.1	-4.58877252983138\\
7.2	-4.48379208167074\\
7.3	-4.36760448841969\\
7.4	-4.24050015855204\\
7.5	-4.1027967866978\\
7.6	-3.95483855957208\\
7.7	-3.79699529568754\\
7.8	-3.6296615210007\\
7.9	-3.45325548280254\\
8	-3.26821810431806\\
8.1	-3.07501188262787\\
8.2	-2.87411973266634\\
8.3	-2.66604378018577\\
8.4	-2.4513041067035\\
8.5	-2.23043744956896\\
8.6	-2.00399586039988\\
8.7	-1.77254532524066\\
8.8	-1.5366643498921\\
8.9	-1.29694251394813\\
9	-1.0539789971539\\
9.1	-0.808381081768432\\
9.2	-0.56076263467527\\
9.3	-0.311742573034958\\
9.4	-0.0619433173144528\\
9.5	0.188010764439883\\
9.6	0.437494917197236\\
9.7	0.685885560504541\\
9.8	0.932561847112879\\
9.9	1.17690721477226\\
10	1.41831092731613\\
10.1	1.65616960118377\\
10.2	1.8898887135649\\
10.3	2.11888408839714\\
10.4	2.34258335650188\\
10.5	2.5604273862092\\
10.6	2.7718716808958\\
10.7	2.97638773994303\\
10.8	3.17346437971317\\
10.9	3.36260901124233\\
11	3.5433488714563\\
11.1	3.71523220483205\\
11.2	3.87782939255125\\
11.3	4.03073402632358\\
11.4	4.1735639241958\\
11.5	4.3059620858076\\
11.6	4.42759758470659\\
11.7	4.53816639549207\\
11.8	4.63739215372018\\
11.9	4.72502684667114\\
12	4.80085143325183\\
12.1	4.86467639148449\\
12.2	4.91634219221292\\
12.3	4.95571969784235\\
12.4	4.98271048511609\\
12.5	4.9972470911225\\
12.6	4.99929318191708\\
12.7	4.98884364333842\\
12.8	4.96592459379096\\
12.9	4.93059331896256\\
13	4.88293812864012\\
13.1	4.82307813598109\\
13.2	4.75116295979265\\
13.3	4.66737235056256\\
13.4	4.5719157411766\\
13.5	4.46503172344538\\
13.6	4.34698745174913\\
13.7	4.2180779752908\\
13.8	4.07862550062678\\
13.9	3.92897858631831\\
14	3.76951127171652\\
14.1	3.60062214205897\\
14.2	3.42273333221403\\
14.3	3.23628947156362\\
14.4	3.04175657266128\\
14.5	2.83962086644347\\
14.6	2.63038758690553\\
14.7	2.41457970827969\\
14.8	2.19273663787195\\
14.9	1.96541286782471\\
15	1.73317658917513\\
15.1	1.49660827167354\\
15.2	1.25629921291128\\
15.3	1.01285006038472\\
15.4	0.766869310189326\\
15.5	0.518971786096265\\
15.6	0.269777102813249\\
15.7	0.0199081172703987\\
15.8	-0.230010628197685\\
15.9	-0.479354466882485\\
16	-0.727500169043068\\
16.1	-0.973827499655807\\
16.2	-1.21772076867896\\
16.3	-1.45857036995713\\
16.4	-1.69577430491917\\
16.5	-1.92873968726111\\
16.6	-2.1568842248531\\
16.7	-2.37963767516655\\
16.8	-2.59644327058343\\
16.9	-2.80675911002536\\
17	-3.01005951342412\\
17.1	-3.20583633564801\\
17.2	-3.39360023660006\\
17.3	-3.57288190431346\\
17.4	-3.74323322798699\\
17.5	-3.90422841802875\\
17.6	-4.05546507030828\\
17.7	-4.19656517195742\\
17.8	-4.32717604620556\\
17.9	-4.44697123388791\\
18	-4.55565130942338\\
18.1	-4.65294462922264\\
18.2	-4.73860801065556\\
18.3	-4.81242733988119\\
18.4	-4.87421810702082\\
18.5	-4.92382586733662\\
18.6	-4.96112662726302\\
18.7	-4.98602715432606\\
18.8	-4.99846521017603\\
18.9	-4.99840970615092\\
19	-4.98586078098189\\
19.1	-4.9608498004465\\
19.2	-4.92343927897064\\
19.3	-4.87372272337495\\
19.4	-4.81182439915655\\
19.5	-4.73789901988997\\
19.6	-4.65213136052377\\
19.7	-4.55473579553944\\
19.8	-4.4459557631268\\
19.9	-4.32606315671536\\
20	-4.19535764538226\\
};
\addlegendentry{${x(t)=x}_{0}{*cos(}\omega{*t)}$}

\addplot [color=black, dashdotted]
  table[row sep=crcr]{%
0	-0\\
0.1	-0.124947923176696\\
0.2	-0.24958354161707\\
0.3	-0.373595331183998\\
0.4	-0.496673326987653\\
0.5	-0.618509898136307\\
0.6	-0.738800516653349\\
0.7	-0.857244518638628\\
0.8	-0.973545855771626\\
0.9	-1.08741383527808\\
1	-1.19856384651051\\
1.1	-1.30671807232665\\
1.2	-1.41160618348759\\
1.3	-1.5129660143401\\
1.4	-1.61054421809423\\
1.5	-1.70409690005834\\
1.6	-1.79339022724881\\
1.7	-1.87820101285073\\
1.8	-1.95831727406871\\
1.9	-2.03353876197343\\
2	-2.10367746201974\\
2.1	-2.16855806398504\\
2.2	-2.22801840015359\\
2.3	-2.2819098506513\\
2.4	-2.33009771491807\\
2.5	-2.37246154838897\\
2.6	-2.40889546354298\\
2.7	-2.43930839456665\\
2.8	-2.46362432497115\\
2.9	-2.48178247759397\\
3	-2.49373746651014\\
3.1	-2.49945941047339\\
3.2	-2.49893400760376\\
3.3	-2.4921625711348\\
3.4	-2.47916202613117\\
3.5	-2.45996486718484\\
3.6	-2.43461907719549\\
3.7	-2.40318800743825\\
3.8	-2.36575021921854\\
3.9	-2.32239928750967\\
4	-2.2732435670642\\
4.1	-2.21840592158344\\
4.2	-2.15802341662218\\
4.3	-2.09224697699624\\
4.4	-2.02124100954898\\
4.5	-1.9451829922198\\
4.6	-1.8642630304418\\
4.7	-1.77868338197711\\
4.8	-1.68865795137788\\
4.9	-1.59441175533626\\
5	-1.49618036025989\\
5.1	-1.39420929347854\\
5.2	-1.28875342955366\\
5.3	-1.18007635322471\\
5.4	-1.06844970058457\\
5.5	-0.954152480130829\\
5.6	-0.837470375389762\\
5.7	-0.718695030856361\\
5.8	-0.598123323034955\\
5.9	-0.476056618402568\\
6	-0.352800020149668\\
6.1	-0.228661605581092\\
6.2	-0.103951656083226\\
6.3	0.0210181184178727\\
6.4	0.14593535856895\\
6.5	0.270487836325271\\
6.6	0.394364235358122\\
6.7	0.517254929183499\\
6.8	0.638852755067079\\
6.9	0.758853781771073\\
7	0.87695806922405\\
7.1	0.992870418214901\\
7.2	1.10630110823713\\
7.3	1.21696662163925\\
7.4	1.32459035227123\\
7.5	1.42890329685586\\
7.6	1.5296447273568\\
7.7	1.62656284266292\\
7.8	1.71941539795994\\
7.9	1.80797031021628\\
8	1.89200623826982\\
8.1	1.97131313606549\\
8.2	2.04569277766103\\
8.3	2.11495925268862\\
8.4	2.17893943103397\\
8.5	2.23747339557146\\
8.6	2.29041484187364\\
8.7	2.33763144389612\\
8.8	2.37900518472379\\
8.9	2.4144326515516\\
9	2.44382529416274\\
9.1	2.46710964625809\\
9.2	2.48422750908366\\
9.3	2.4951360968972\\
9.4	2.49980814391025\\
9.5	2.49823197243845\\
9.6	2.4904115220896\\
9.7	2.47636633991678\\
9.8	2.45613153156083\\
9.9	2.42975767350455\\
10	2.39731068665785\\
10.1	2.35887167158977\\
10.2	2.31453670581933\\
10.3	2.26441660367176\\
10.4	2.20863663930038\\
10.5	2.14733623356648\\
10.6	2.08066860555975\\
10.7	2.00880038963039\\
10.8	1.93191121888997\\
10.9	1.85019327622224\\
11	1.76385081392598\\
11.1	1.67309964319066\\
11.2	1.5781665946808\\
11.3	1.47928895157753\\
11.4	1.37671385649409\\
11.5	1.27069769374815\\
11.6	1.16150544853439\\
11.7	1.04941004459965\\
11.8	0.934691662075592\\
11.9	0.817637037174352\\
12	0.698538745497315\\
12.1	0.577694470748481\\
12.2	0.45540626068024\\
12.3	0.331979772131292\\
12.4	0.207723507043743\\
12.5	0.082948041368892\\
12.6	-0.0420347512108743\\
12.7	-0.166912478803889\\
12.8	-0.291373012126234\\
12.9	-0.415105264662391\\
13	-0.537799970219539\\
13.1	-0.659150455931946\\
13.2	-0.778853408783445\\
13.3	-0.896609633732001\\
13.4	-1.01212480154149\\
13.5	-1.12511018445154\\
13.6	-1.23528337784652\\
13.7	-1.34236900612003\\
13.8	-1.4460994109705\\
13.9	-1.54621532040756\\
14	-1.64246649679697\\
14.1	-1.73461236232441\\
14.2	-1.82242260031469\\
14.3	-1.90567773090353\\
14.4	-1.98416965962288\\
14.5	-2.05770219752876\\
14.6	-2.12609155157141\\
14.7	-2.18916678398221\\
14.8	-2.24677023952907\\
14.9	-2.29875793957243\\
15	-2.34499994193685\\
15.1	-2.38538066569879\\
15.2	-2.41979918007872\\
15.3	-2.4481694567155\\
15.4	-2.4704205846925\\
15.5	-2.48649694777794\\
15.6	-2.49635836343651\\
15.7	-2.4999801832648\\
15.8	-2.49735335459943\\
15.9	-2.48848444314405\\
16	-2.47339561655845\\
16.1	-2.452124589051\\
16.2	-2.42472452711272\\
16.3	-2.39126391662877\\
16.4	-2.35182639169943\\
16.5	-2.30651052559835\\
16.6	-2.25542958439073\\
16.7	-2.19871124382716\\
16.8	-2.1364972702207\\
16.9	-2.06894316610496\\
17	-1.99621778155873\\
17.1	-1.91850289216872\\
17.2	-1.83599274468528\\
17.3	-1.74889357150667\\
17.4	-1.65742307520546\\
17.5	-1.56180988438548\\
17.6	-1.4622929822294\\
17.7	-1.35912110916522\\
17.8	-1.25255214114471\\
17.9	-1.14285244508789\\
18	-1.03029621310439\\
18.1	-0.915164777156028\\
18.2	-0.79774590587338\\
18.3	-0.678333085284082\\
18.4	-0.557224785250619\\
18.5	-0.434723713451084\\
18.6	-0.311136058767654\\
18.7	-0.186770725973837\\
18.8	-0.0619385636333944\\
18.9	0.0630484128591468\\
19	0.187877801154523\\
19.1	0.312237592791881\\
19.2	0.435816953057449\\
19.3	0.55830699790946\\
19.4	0.679401566027356\\
19.5	0.798797984055684\\
19.6	0.916197823129821\\
19.7	1.03130764479273\\
19.8	1.1438397344383\\
19.9	1.25351282044799\\
20	1.36005277722342\\
};
\addlegendentry{${v(t)=-}\omega{*x}_{0}{*sin(}\omega{*t)}$}

\addplot [color=black, dotted]
  table[row sep=crcr]{%
0	-1.25\\
0.1	-1.24843782549371\\
0.2	-1.24375520659753\\
0.3	-1.23596384742005\\
0.4	-1.22508322230155\\
0.5	-1.21114052713831\\
0.6	-1.19417061140701\\
0.7	-1.17421589105922\\
0.8	-1.15132624250361\\
0.9	-1.12555887794085\\
1	-1.09697820236297\\
1.1	-1.06565565257438\\
1.2	-1.0316695186371\\
1.3	-0.99510474818632\\
1.4	-0.95605273410561\\
1.5	-0.914611086092276\\
1.6	-0.870883386683957\\
1.7	-0.824978932356228\\
1.8	-0.77701246033833\\
1.9	-0.727103861829854\\
2	-0.675377882335175\\
2.1	-0.621963809864659\\
2.2	-0.566995151781972\\
2.3	-0.510609301105196\\
2.4	-0.452947193095842\\
2.5	-0.394152952994086\\
2.6	-0.334373535780734\\
2.7	-0.273758358866302\\
2.8	-0.212458928625301\\
2.9	-0.150628461709208\\
3	-0.0884215020846286\\
3.1	-0.0259935347538655\\
3.2	0.036499402876611\\
3.3	0.0989011110084176\\
3.4	0.161055617869406\\
3.5	0.222807569561865\\
3.6	0.284002618366359\\
3.7	0.344487808530641\\
3.8	0.404111958579379\\
3.9	0.462726039189109\\
4	0.520183545683928\\
4.1	0.576340864220891\\
4.2	0.631057630749822\\
4.3	0.684197081850339\\
4.4	0.735626396569182\\
4.5	0.785217028403424\\
4.6	0.832845026599781\\
4.7	0.878391345966942\\
4.8	0.921742144426557\\
4.9	0.962789067559134\\
5	1.00142951943367\\
5.1	1.03756691904403\\
5.2	1.07111094171118\\
5.3	1.10197774484786\\
5.4	1.13009017752133\\
5.5	1.15537797329058\\
5.6	1.17777792583582\\
5.7	1.19723404694136\\
5.8	1.21369770643699\\
5.9	1.22712775374807\\
6	1.23749062075056\\
6.1	1.24476040567383\\
6.2	1.2489189378416\\
6.3	1.24995582308918\\
6.4	1.24786846974344\\
6.5	1.24266209510068\\
6.6	1.23434971238608\\
6.7	1.22295209822742\\
6.8	1.20849774072433\\
6.9	1.19102276824288\\
7	1.1705708591135\\
7.1	1.14719313245784\\
7.2	1.12094802041768\\
7.3	1.09190112210492\\
7.4	1.06012503963801\\
7.5	1.02569919667445\\
7.6	0.988709639893021\\
7.7	0.949248823921885\\
7.8	0.907415380250175\\
7.9	0.863313870700634\\
8	0.817054526079515\\
8.1	0.768752970656968\\
8.2	0.718529933166586\\
8.3	0.666510945046443\\
8.4	0.612826026675874\\
8.5	0.557609362392241\\
8.6	0.500998965099969\\
8.7	0.443136331310164\\
8.8	0.384166087473024\\
8.9	0.324235628487033\\
9	0.263494749288475\\
9.1	0.202095270442108\\
9.2	0.140190658668818\\
9.3	0.0779356432587396\\
9.4	0.0154858293286132\\
9.5	-0.0470026911099707\\
9.6	-0.109373729299309\\
9.7	-0.171471390126135\\
9.8	-0.23314046177822\\
9.9	-0.294226803693064\\
10	-0.354577731829033\\
10.1	-0.414042400295942\\
10.2	-0.472472178391225\\
10.3	-0.529721022099284\\
10.4	-0.58564583912547\\
10.5	-0.640106846552301\\
10.6	-0.692967920223951\\
10.7	-0.744096934985758\\
10.8	-0.793366094928292\\
10.9	-0.840652252810582\\
11	-0.885837217864075\\
11.1	-0.928808051208012\\
11.2	-0.969457348137812\\
11.3	-1.00768350658089\\
11.4	-1.04339098104895\\
11.5	-1.0764905214519\\
11.6	-1.10689939617665\\
11.7	-1.13454159887302\\
11.8	-1.15934803843004\\
11.9	-1.18125671166778\\
12	-1.20021285831296\\
12.1	-1.21616909787112\\
12.2	-1.22908554805323\\
12.3	-1.23892992446059\\
12.4	-1.24567762127902\\
12.5	-1.24931177278062\\
12.6	-1.24982329547927\\
12.7	-1.24721091083461\\
12.8	-1.24148114844774\\
12.9	-1.23264832974064\\
13	-1.22073453216003\\
13.1	-1.20576953399527\\
13.2	-1.18779073994816\\
13.3	-1.16684308764064\\
13.4	-1.14297893529415\\
13.5	-1.11625793086135\\
13.6	-1.08674686293728\\
13.7	-1.0545194938227\\
13.8	-1.0196563751567\\
13.9	-0.982244646579577\\
14	-0.942377817929131\\
14.1	-0.900155535514743\\
14.2	-0.855683333053508\\
14.3	-0.809072367890905\\
14.4	-0.760439143165319\\
14.5	-0.709905216610869\\
14.6	-0.657596896726382\\
14.7	-0.603644927069922\\
14.8	-0.548184159467988\\
14.9	-0.491353216956177\\
15	-0.433294147293782\\
15.1	-0.374152067918384\\
15.2	-0.31407480322782\\
15.3	-0.25321251509618\\
15.4	-0.191717327547332\\
15.5	-0.129742946524066\\
15.6	-0.0674442757033122\\
15.7	-0.00497702931759967\\
15.8	0.0575026570494212\\
15.9	0.119838616720621\\
16	0.181875042260767\\
16.1	0.243456874913952\\
16.2	0.304430192169739\\
16.3	0.364642592489283\\
16.4	0.423943576229793\\
16.5	0.482184921815277\\
16.6	0.539221056213276\\
16.7	0.594909418791637\\
16.8	0.649110817645857\\
16.9	0.701689777506339\\
17	0.75251487835603\\
17.1	0.801459083912002\\
17.2	0.848400059150016\\
17.3	0.893220476078365\\
17.4	0.935808306996748\\
17.5	0.976057104507186\\
17.6	1.01386626757707\\
17.7	1.04914129298935\\
17.8	1.08179401155139\\
17.9	1.11174280847198\\
18	1.13891282735585\\
18.1	1.16323615730566\\
18.2	1.18465200266389\\
18.3	1.2031068349703\\
18.4	1.2185545267552\\
18.5	1.23095646683415\\
18.6	1.24028165681575\\
18.7	1.24650678858151\\
18.8	1.24961630254401\\
18.9	1.24960242653773\\
19	1.24646519524547\\
19.1	1.24021245011163\\
19.2	1.23085981974266\\
19.3	1.21843068084374\\
19.4	1.20295609978914\\
19.5	1.18447475497249\\
19.6	1.16303284013094\\
19.7	1.13868394888486\\
19.8	1.1114889407817\\
19.9	1.08151578917884\\
20	1.04883941134557\\
};
\addlegendentry{${a(t)=-}\omega{}^{2}{*x}_{0}{*cos(}\omega{*t)}$}
\end{axis}
\end{tikzpicture}
\caption{Die Harmonische Schwingung}
\end{figure}
\subsection{Unterkapitel im Anhang}
\lipsum[1-6]
\section{Zweiter Anhang}
\lipsum[1-6]

%***********************************************************
\end{document}