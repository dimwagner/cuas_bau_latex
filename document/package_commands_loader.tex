% Package- and Custom-Commands-Loader

% Paketeinbindung
\usepackage[english, ngerman]{babel}% Sprachpakete für Silbentrennung
\let\latinencoding\relax
\usepackage{csquotes}
\usepackage{fontspec}% Paket zur Einbindung von Schriftarten
\usepackage[headsepline]{scrlayer-scrpage}% Paket für Kopf und Fußzeile
\usepackage{geometry}
\usepackage{titletoc}
\usepackage{lipsum}

% Grafiken
% \usepackage{transparent}
\usepackage{tikz}
\usepackage{pgfplots}
\usepackage{rotating}
% \usepackage{pgf-pie} % Für Die erstellung von Tortendiagrammen, Kompatibilitätsproblem auf Overleaf
\usepackage{xstring}
\usepackage{calc}
\usepackage{subfiles}
\usepackage{eqparbox}
\usepackage{appendix}% Anhangpaket
\usepackage{amsmath,amssymb,amstext}% Mathematische Symbolik
\usepackage{mathtools}
\usepackage{graphicx}
\usepackage{array}
\usepackage{filecontents}
\usepackage{datatool}
\usepackage{tabularx}% Für Tabellen bei denen die X-Umgebung gebraucht wird
\usepackage{booktabs}% schönere Tabellen-Linien
\usepackage{float}
\usepackage{pdfpages}% Einbinden von PDF
\usepackage[%
format=hang, 
labelsep=space, 
labelsep=colon]{caption}% Anpassung der Unterschriften (Bilder/Tabellen)
\usepackage{soul}% Hervorhebung von Text
\usepackage{chemformula}% Für die chemische Formel
\usepackage[onehalfspacing]{setspace}% Anderthalber Zeilenabstand  


\usepackage[pdfencoding=auto,hidelinks,]{hyperref}% Link-Paket für Kapitelverlinkung. Als letztes laden!!

% Bibliotheken
\usetikzlibrary{%
	backgrounds,							% Hintergrund
	calc,									% Berechnungen
	positioning,							% relative Positionierungen
	chains, 
	shapes, 
	trees, 
	fadings, 
	external,
	intersections,
	through,
	decorations.pathmorphing,
	arrows,
}
\pgfplotsset{compat=1.15}

% Custom-Commands
\newlength{\tocindent}
\newcommand{\inhaltsverzeichnis}{}
\newcommand{\anhangtrenner}{}
\newcommand{\abbildungsverzeichnis}{}
\newcommand{\tabellenverzeichnis}{}
\newcommand{\anhangverzeichnis}{}
\newcommand{\literaturverzeichnis}{}
\newcommand{\isincluded}{}
\newcommand{\arbeit}{}
\newenvironment{conditions}{}{}
\renewcommand{\arbeit}{}
\newcommand{\normenverzeichnis}{}
\newcommand{\abkuerzungsverzeichnis}{}