\section{Mathematische Formel}
\label{sec:mathematischeformel}
Nachfolgend sehen wir uns verschiedene Formeln an
\subsection{Eigenständige Formel mit Erklärung}
\label{sec:eq_mit_erklaerung}
Die Erklärung beschreibt die in der Formel verwendeten Zeichen
\begin{equation}
\delta = \frac{C}{\lambda}\left(\sigma_{1}-\sigma_{2}\right)d
\label{eq:HG_Spannungsoptik}
\end{equation}
\begin{conditions}
	\sigma_{1,2} & Hauptspannungen [$kN/mm^{2}$]\\
	d & Modellstärke [$mm$]\\
	\delta & Phasenverschiebung [$-$]\\
	C & Materialkonstante [$mm^{2}/kN$]\\
	\lambda & Wellenlänge des Lichts im Vakuum [$nm$]
\end{conditions}

\subsection{Formelbündelung}
\label{sec:formelbuendel}
Formeln können als Subequations gebündelt werden. Die Umgebung "`align"' hilft bei der Ausrichtung
\begin{subequations} \label{grp:phi}
	\begin{align}
	\varphi_{\text{\tiny I}} &= \varphi_{i+1} \label{eq:phi_1}\\
	\varphi_{\text{\tiny II}} &= \varphi_{i+1}+\frac{1}{2}\pi \label{eq:phi_2}\\
	\varphi_{\text{\tiny III}} &= \varphi_{i+1}+\pi \label{eq:phi_3}\\
	\varphi_{\text{\tiny  IV}} &= \varphi_{i+1}+\frac{3}{2}\pi \label{eq:phi_4}
	\end{align}
\end{subequations}

\subsection{Mathematikmodus im Text}
\label{sec:textmathe}
Man kann auch den Mathematikmodus im Text selbst einschalten, wie nachfolgend zu sehen: $\pi \approx 3.141593$. So kann man unter anderem auch auf griechische Buchstaben verweisen.
\clearpage