\section{Was ist LaTeX und wofür ist es gut?
	\label{sec:latex}}
{\LaTeX} (sprich "`Latech"') ist ein Softwarepaket, das die Benutzung des Textsatzsystems TeX mit Hilfe von Makros vereinfacht. LaTeX liegt derzeit in der Version $2_\varepsilon$ vor.

Im Gegensatz zu anderen Textverarbeitungsprogrammen, die nach dem What-you-see-is-what-you-get-Prinzip funktionieren, arbeitet der Autor mit Textdateien, in denen er innerhalb eines Textes anders zu formatierende Passagen oder Überschriften mit Befehlen textuell auszeichnet. Das Beispiel unten zeigt den Quellcode eines einfachen LaTeX-Dokuments. Bevor das LaTeX-System den Text entsprechend setzen kann, muss es den Quellcode verarbeiten. Das dabei von LaTeX generierte Layout gilt als sehr sauber, sein Formelsatz als sehr ausgereift. Außerdem ist die Ausgabe u. a. nach PDF, HTML und PostScript möglich. LaTeX eignet sich insbesondere für umfangreiche Arbeiten wie Diplomarbeiten und Dissertationen, die oftmals strengen typographischen Ansprüchen genügen müssen. Insbesondere in der Mathematik und den Naturwissenschaften erleichtert LaTeX das Anfertigen von Schriftstücken durch seine komfortablen Möglichkeiten der Formelsetzung gegenüber üblichen Textverarbeitungssystemen. Das Verfahren von LaTeX wird auch mit WYSIWYAF (What you see is what you asked for.) umschrieben.

Das schrittweise Arbeiten erfordert vordergründig im Vergleich zu herkömmlichen Textverarbeitungen einerseits eine längere Einarbeitungszeit, andererseits kann das Aussehen des Resultats genau festgelegt werden. Die längere Einarbeitungszeit kann sich jedoch, insbesondere bei Folgeprojekten mit vergleichbarem Umfang oder ähnlichen Erfordernissen, lohnen.[8] Inzwischen gibt es auch grafische Editoren, die mit LaTeX arbeiten können und WYSIWYG oder WYSIWYM (What you see is what you mean.) bieten und ungeübten Usern den Einstieg deutlich erleichtern können. Beispiele für LaTeX-Entwicklungsumgebungen sind im Abschnitt Entwicklungsumgebungen aufgelistet.\footnote{Quelle: \href{https://de.wikipedia.org/wiki/LaTeX}{wikipedia.org}}

\clearpage

\subsection{Was brauche ich für Programme und was kostet der Spaß?
	\label{sec:latex_programme}}
Zum einen benötigt man eine TeX-Distribution, zum anderen einen Editor. hier reicht theoretisch schon der in Windows verbaute \textbf{notepad}. Es gibt allerdings Editoren, die speziell auf {\LaTeX} ausgerichtet sind. Zur Quellenverwaltung kann Citavi oder die kostenlose Alternative "`JabRef"' Verwendung finden.

\begin{center}
	\begin{tabular}{ccc}
	\href{https://miktex.org/}{Tex-Distribution MiKTeX} &
	\href{https://www.texstudio.org/}{Editor TeXstudio} &
	\href{http://www.jabref.org/}{JabRef} \\
\end{tabular}
\end{center}

Bei der Installation kann man eigentlich kaum etwas falsch machen. Bei MiKTeX empfiehlt es sich die Option "`Install packages on the fly"' zu aktivieren. So erspart man sich das manuelle Laden von Paketen. 

{\LaTeX} wie auch die meiste benötigte Software sind Open Source, das heißt, dass hier keine Kosten anfallen.

\subsection{Wo finde ich Hilfe wenn ich etwas brauche? 
	\label{sec:latex_hilfe}}
\begin{itemize}
	\item \href{http://projekte.dante.de/DanteFAQ/WebHome}{Projekt dante - \LaTeX -FAQ}
	\item \href{https://tex.stackexchange.com/}{TeX-Stackexchange}
\end{itemize}
\textcolor{red}{Am besten nun noch die Starteinstellungen "`Startprofil.txsprofile"' über die Optionen laden.}
\cleardoublepage
